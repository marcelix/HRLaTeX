\documentclass[openany, a4paper]{memoir}
\usepackage[croatian]{babel}
%\usepackage[Rejne]{fncychap}
\usepackage[cp1250]{inputenc}
\usepackage[linkcolor=blue, urlcolor=blue, colorlinks]{hyperref}
\usepackage{titlesec}
\usepackage[a4paper,left=2.2cm,top=2cm,right=1.6cm,
bottom=2cm%
%,showframe
]{geometry}

\usepackage{palatino}

\usepackage{titletoc}


% Essential packages
\usepackage{amsmath}
\usepackage{enumerate}
\usepackage[draft]{fixme}
\usepackage[dvips]{graphicx}
\usepackage{rotating}
\usepackage{multido}

\renewcommand{\thechapter}{\Roman{chapter}}



\setcounter{tocdepth}{4}
\setcounter{secnumdepth}{4}




\title{HR\LaTeX}

\begin{document}


%\multiput(x coord,y coord)(delta x,delta y){number of copies}{object}

%\maketitle

\marginpar{\rotatebox{90}{\Huge  HR\LaTeX}}



  \tableofcontents
  \newpage
  {\small \listoffixmes }




  \chapter{Instalacija}
  \section{Windows}
  \subsection{\TeX distribucije}
  \href{http://www.miktex.org/}{MikTeX}
%
  \subsection{Editori}
  Komercijalno (WinEdt) ili \emph{free} (TeXnicCenter).

  \subsection{Ghostscript}


  \section{Linux/Mac}
  \subsection{tetex}


  \section{\LaTeX$\xrightarrow{\ \text{pdflatex}\ }$PDF}
  \section{\LaTeX$\xrightarrow{\ \text{dvips}\    }$PS$\xrightarrow{\ \text{ps2pdf}\ }$PDF}




  \chapter{Primjeri, uzorci dokumenata}


  \section{Knjige}
    Za podlogu uzeti ``Diferencijalne jednad�be'' od Z.~�iki�a (ta je najsre�enija).
    Za kasnije spajanje u individualne class datoteke treba slo�iti da se
    raskomadana preambula spoji u jednu datoteku i da se to trpa u klasu.

  \subsection{Ud�benik}
  Sveu�ili�ni ud�benik

  \subsection{memoir klasa}

  \subsection{Zbirka zadataka}

  \subsection{Doktorska disertacija}

  \subsection{Magistarski rad}

  \subsection{Diplomski rad}

  \section{�lanci}
  \subsection{article}
  \subsection{amsart}

  \section{Slajdovi}


  \section{Osnovno}
    \subsection{babel}
    \begin{verbatim}
        \usepackage[babel]{croatian}
    \end{verbatim}

    \subsection{inputenc}
    \begin{verbatim}
        \usepackage[cp1250]{inputenc}
    \end{verbatim}

%============================================================

  \chapter{``New Document Templates'' za programe editore}
  \section{WinEdt}

  \section{\TeX niccenter}
    Ima \emph{feature} ``Document $\rightarrow$ new from template''.








  \chapter{WWW}



  \section{Sadr�aji na webu}


    \subsection{Tutoriali}
        \begin{itemize}
            \item  instalacija za win32, linux
            \item  grafika
            \begin{itemize}
              \item crtanje
              \begin{itemize}
                \item vektorske slike (eps format iz npr. Corel Draw, Illustrator, xfig itd.)
                \item rasterski formati (bitmape)
              \end{itemize}
              \item Mathematica
              \item \texttt{psfrag} paket (s slikama iz Mathematice)
              \item pstricks
            \end{itemize}
            \item  fontovi
            \item  izrada kazala (makeindex) !!!\\
            (stilovi, zamrzavanje indeksa, ru�no dovr�avanje,...)
            \item  najkorisniji paketi:
            babel, inputenc, amsmath, hyperref, geometry, layout, crop, fixme,   \fxnote{shorttoc}
            \fxnote{grupirati neke u ``napredniju'' grupu}

            \item  ``provjera znanja'' -- jednostavan dokument, cijeli dokumentiran..
            (tilda u paragrafu, itd.)
        \end{itemize}

    \subsection{Novosti - preko bloga}
        Nove verzije programa (editori, \TeX distribucija, Ghostscript, GSView, \dots)
        Nove dobre knjige, dobri tutoriali na webu, linkovi...

    \newpage
    \subsection{FAQ - �esta pitanja}

        To mo�emo na kraju...

        Skica FAQ-a (popis tema):
        \begin{enumerate}
        \item Detaljan popis simbola za \LaTeX (comprehensive \LaTeX symbol list)
        \item Komotnije tablice
        \texttt{%
        $\backslash$renewcommand\{$\backslash$arraystretch\}\{1.15\}
        }
        \begin{verbatim}
            \renewcommand{\arraystretch}{1.15}
        \end{verbatim}
        \end{enumerate}


        Napredno:
        \begin{enumerate}
          \item TDS - (\emph{eng.}~\TeX directory structure)
        \end{enumerate}




    \newpage
    \section{Tehni�ki detalji}
        \subsection{SourceForge.net hrlatex}
        Registrirano.

        \subsection{Domena}
        Registracija www.tug.hr domene.

        \subsection{Tehni�ka pozadina (software)}

        \subsubsection{WordPress}
        Instalirati WordPress 1.5.x s LetterHead temom (podsjeca na tiskane knjige)
        \subsubsection{MediaWiki wiki}
            mediawiki (oprobano), radi na \texttt{titan.fsb.hr} stroju


        \subsubsection{RSS - via WordPress}




  \chapter{Lokalizacija}

    \section{Lokalizacije paketa}

    \begin{itemize}
        \item fixme.sty
        \item ??
    \end{itemize}






  \chapter{Nabava}
    \section{Ra�unalo}
        \subsection{opis}
        AMD64 PC ra�unalo s 19" LCD monitorom + laserski Lexmark printer


    \section{Knjige}
        \begin{enumerate}
            \item H. Kopka, P. Daly: Guide to \LaTeX
        \end{enumerate}
        Naru�eno 27.05.2005.



  \appendix

  \chapter{About}


  \fxnote{titlesec, titletoc}
  \fxnote{Prijevod ``List of Corrections''}

\end{document}
