\documentclass{ispit}


\title{Matematika 3}

\begin{document}

\begin{ispit}{06.~srpnja 2006.}

    \begin{napomena}
    Matematika 3A rje\v{s}ava zadatke 1--6; Matematika 3  rje\v{s}ava 1,3, 4, 6, 7, 9;
    Matematika 3B rje\v{s}ava zadatke~4--9.
    \end{napomena}

    \bigskip


%>>>>>>>>>>>>>>>>>>>>>.

\begin{zadatak}
	Funkciju $\sin x + \sh x$ aproksimiramo u ra\v{c}unalu,
	kori\v{s}tenjem po\v{c}etnih komada Taylorovih redova oko $0$ za te
	funkcije. \v{C}lanove svakog reda zbrajamo sve dok prvi odba\v{c}eni
	\v{c}lan ne padne ispod zadane to\v{c}nosti $\varepsilon$,
	$0 < \varepsilon \ll 1$. Ho\'{c}e li za $x = 10$ takva aproksimacija
	biti pribli\v{z}no to\v{c}na ili ne? Objasnite.
\end{zadatak}

\begin{zadatak}
	Profesor Senilkovi\'{c} na\v{s}ao se u problemima, jer je
	zaboravio je li LR faktorizaciju matrice radio s parcijalnim
	pivotiranjem ili bez njega. Dobivena matrica $L$ bila je
	\begin{displaymath}
		L = \left[
			\begin{array}{rrr}
			 1 &   &   \\
			 2 & 1 &   \\
			 0 & 0 & 1
			\end{array}
			\right].
	\end{displaymath}
	Pomozite prof.\ Senilkovi\'{c}u i objasnite mu zbog \v{c}ega je
	odmah vidljivo je li koristio pivotiranje ili ne. 
\end{zadatak}

\begin{zadatak}
	Na\dj ite koliko je podintervala potrebno (po ocjeni gre\v{s}ke),
	a zatim produljenom Simpsonovom metodom izra\v{c}unajte
	pribli\v{z}nu vrijednost integrala
	\begin{displaymath}
		\int_1^2 \left( \frac{x^5}{60} + \frac{x^4}{4} + 2x^2
				- x \right) \, dx
	\end{displaymath}
	tako da gre\v{s}ka bude manja od $10^{-4}$.
\end{zadatak}

%>>>>>>>>>>>>>>>>>>>>>.
 
\begin{zadatak}
Bacamo dvije igra\'ce kocke. Dobiveni zbroj na njima je 8. Koja je vjerojatnost da je na jednoj od njih (svejedno kojoj) pala 2-ojka?
\end{zadatak}

\begin{zadatak}
Za slu\v{c}ajnu varijablu $X$ koja ima normalnu razdiobu
$\mathcal{N}(\mu =100, \sigma)$ odredite standardnu devijaciju $\sigma$ tako da vrijedi
$$P(99 < X < 101)=0.5 .$$
\end{zadatak}

    \begin{zadatak}
    Predsjedni\v{c}ki kandidat $A$ pobijediti \'ce kandidata $B$ na predsjedni\v{c}kim izborima sa $55\%$ glasova.
\begin{compactenum}[a)]
\item Kolika je vjerojatnost da slu\v{c}ajni uzorak veli\v{c}ine $N=200$ glasa\v{c}a predvidi krivi ishod izbora -- odnosno da u tom uzorku $A$ dobije manje od $50\%$ glasova?
\item Kolika mora biti veli\v{c}ina uzorka da vjerojatnost krive prognoze izbora bude manja od $5\%$?
\end{compactenum}
    \end{zadatak}

\begin{zadatak}
     Polo\v{z}aji dviju \v{c}estica koje se gibaju u prostoru  dani su parametrizacijama
      $$\vec{r}_1(t) = (2t^3, 1-t, t^2), \qquad \vec{r}_2 = (1+t, t^2+2, t^3)\ .$$
\begin{compactenum}[a)]
\item Kolika je me\dj{}usobna udaljenost \v{c}estica u trenutku $t=0$?
\item Koja \v{c}estica ima ve\'ce ubrzanje u trenutku $t=1$?
\end{compactenum}
\end{zadatak}

\begin{zadatak}
  Neka je $U$ skalarno polje zadano s
    $ U = xyz +  yz + z-1$.
  Izra\v cunajte $$\int_{K} U |d\overrightarrow{r}|$$ gdje je $K$
  du\v zina koja spaja to\v cke $A(3,0,1)$ i $B(3,1,2)$.
\end{zadatak}

\begin{zadatak}
Izra\v{c}unajte
$$\oint_C \vec{F} d\vec{r}$$
za polje $\overrightarrow{F}=(y^2, zy, xy)$. Krivulja $C$ je pozitivno orijentirani rub kvadrata s vrhovima $(1,1)$, $(-1,1)$, $(-1,-1)$, $(1,-1)$ u $xy$-ravnini.
%   \bodovi{20 bodova}
\end{zadatak}

\end{ispit}

\end{document}
