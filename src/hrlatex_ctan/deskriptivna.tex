\documentclass[%
% a4paper,
marcelix,
draft,
% slideColor,
slideBW,
% colorBG,
ps2pdf,
pdf]{prosper}


% \hypersetup{pdfpagemode=FullScreen}%

% \usepackage{enumerate}

\usepackage{hrlatex}


\usepackage{amsmath}

\usepackage{txfonts}




\title{Deskriptivna statistika}
	\subtitle{ opisivanje podataka }
	\author{Marcelix}
%  	\email{marcelix@spam.hr}
%  	\institution{\href{http://www.tug.hr/}{\sffamily \textbf{HR}LaTeX}}





\begin{document}



   \Logo{\footnotesize\sffamily \fontsize{4pt}{14pt}\selectfont \textbf{HR}LaTeX}

\maketitle



% =============================

\begin{slide}{Kako opisati podatke?}
\begin{itemize} 
 \item prvi red
 \item drugi red
\end{itemize}

\end{slide}


\overlays{4}{
\begin{slide}{Drugi slajd}
\untilSlide{3}{{\scriptsize (ovo je tekst koji \'ce nestati s ekrana poslije 3.~linije)}}
\begin{itemstep}
\item $1+1=2$
\item $\displaystyle \int_1^x \frac{1}t dt= \ln x$
\item tre\'ci redak
\item $1+2+3+\dots +n = ?$
\end{itemstep}
\end{slide}
}


\begin{slide}{Kraj}
 
 \bigskip
 
 \hfil
 Hvala.
\end{slide}



\end{document}
