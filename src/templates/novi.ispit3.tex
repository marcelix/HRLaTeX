\documentclass[a4paper,11pt]{article}
\usepackage{amsfonts}
\usepackage{array}
\usepackage{fancybox}
\usepackage{amsmath}
\usepackage[croatian]{babel}

\addtolength{\voffset}{-2cm} \addtolength{\textheight}{3cm}
\addtolength{\textwidth}{2cm} \addtolength{\hoffset}{-2cm}
\pagestyle{empty}

\begin{document}
\newcounter{grupe}

\begin{center}
\shadowbox{Naziv kolegija}\\[5pt]
datum\\[5pt]
-- pismeni ispit --
\end{center}

\vspace*{10pt}

\noindent IME I PREZIME:\, \rule{6cm}{0.5pt}\\[0.5cm]

\noindent\textbf{popunjava profesor:}\\[5pt]
{\setlength{\arrayrulewidth}{0.8pt} \setlength{\extrarowheight}{8pt}
\begin{tabular}{|c|c|c|c|c|c|c!{\vrule width 3pt}c!{\vrule width 3pt}c|}
\hline
\raisebox{2pt}{ZADATAK} & \raisebox{2pt}{1.} & \raisebox{2pt}{2.} & \raisebox{2pt}{3.} & \raisebox{2pt}{4.} & \raisebox{2pt}{5.} & \raisebox{2pt}{6.} & \raisebox{2pt}{UKUPNO} & \raisebox{2pt}{OCJENA} \\
\hline
\raisebox{2pt}{BROJ BODOVA} & & & & & & & & \\
\hline
\end{tabular}
}\\[0.5cm]

\noindent \textbf{Uputa.} U svakoj grupi zadataka treba za prolaznu ocjenu
imati barem 10 bodova i ukupni broj bodova mora biti ve\'{c}i od 30.\\[5pt]

\vspace{15pt}

\noindent\shadowbox{I. grupa}

\begin{enumerate}
\item prvi zadatak
\begin{enumerate}
\item a) dio prvog zadatka
\item b) dio prvog zadatka
\item c) dio prvog zadatka
\end{enumerate}

\marginpar{
\vspace*{-2cm}                %za vertikalno poravnjanje tablice (dizanje i spustanje)
\small                        %velicina slova u tablici
\begin{tabular}{|c|c|}
\hline
(a) & 3 boda\\ \hline
(b) & 4 boda\\ \hline
(c) & 3 boda \\ \hline
\end{tabular}
}

\item drugi zadatak
\begin{enumerate}
\item a) dio drugog zadatka
\item b) dio drugog zadatka
\end{enumerate}

\marginpar{
\vspace*{-1cm}
\small
\begin{tabular}{|c|c|}
\hline
(a) & 5 bodova\\ \hline
(b) & 5 bodova\\ \hline
\end{tabular}
}

\item Odredite domenu, nulto\v{c}ke, ekstreme, intervale monotonosti, asimptote i nacrtajte graf funkcije $$f(x)=\frac{x^2}{4+x^2}\,.$$

\marginpar{
\vspace*{-40pt}
\scriptsize
\begin{tabular}{|c|c|}
\hline
domena & 1 bod\\ \hline
nulto\v{c}ke & 1 bod\\ \hline
int. mon. & 2 boda \\ \hline
ekstremi & 2 boda \\ \hline
asimptote & 2 boda \\ \hline
graf & 2 boda \\ \hline
\end{tabular}
}
\setcounter{grupe}{\value{enumi}}
\end{enumerate}

\vspace*{0.5cm}

\noindent\shadowbox{II. grupa}

\begin{enumerate}
\setcounter{enumi}{\value{grupe}}

\item \v{c}etvrti zadatak
\begin{enumerate}
\item a) dio zadatka
\item b) dio zadatka
\end{enumerate}

\marginpar{
\vspace*{-1.5cm}
\small
\begin{tabular}{|c|c|}
\hline
(a) & 5 bodova\\ \hline
(b) & 5 bodova\\ \hline
\end{tabular}
}

\newpage %ako zelimo ici na novu stranicu

\item peti zadatak
\begin{enumerate}
\item a) dio
\item b) dio
\item c) dio
\end{enumerate}

\marginpar{
\vspace*{-1.5cm}
\small
\begin{tabular}{|c|c|}
\hline
(a) & 2 boda\\ \hline
(b) & 2 boda\\ \hline
(c) & 6 bodova\\ \hline
\end{tabular}
}

\item \v{s}esti zadatak
\begin{enumerate}
\item a) dio zadatka
\item b) dio zadatka
\end{enumerate}
Nastavak \v{s}estog zadatka.

\marginpar{
\vspace*{-1.5cm}
\small
\begin{tabular}{|c|c|}
\hline
(a) & 5 bodova\\ \hline
(b) & 5 bodova\\ \hline
\end{tabular}
}
\end{enumerate}

\end{document}
