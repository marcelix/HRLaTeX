\documentclass[a4paper,12pt]{article}
\usepackage[croatian]{babel}
\newtheorem{teorem}{Teorem}[section]
\newtheorem{propozicija}{Propozicija}[section]
\newtheorem{korolar}{Korolar}[section]
\newtheorem{lema}{Lema}[section]

\begin{document}

\title{naslov prvi red\\
   naslov drugi red}

%------------------autor(i)----------------------------------
\author{ime1\thanks{...}\\
    adresa1 redak1\\
    adresa1 redak2\\
    adresa1 redak3
    \and
    ime2\thanks{...}\\
    adresa2 redak1\\
    adresa2 redak2\\
    adresa2 redak3
    \and
    ime3\thanks{...}\\
    adresa3 redak1\\
    adresa3 redak2\\
    adresa3 redak3}
%-----------------autor(i) kraj------------------------------

\date{unesite datum}
\maketitle

\begin{abstract}
Kratki uvodni tekst
\end{abstract}

%--------prva sekcija i njene podsekcije--------------------
\section{sekcija1}
neki tekst
\subsection{podsekcija1}
neki tekst
\subsubsection{podpodsekcija1}
neki tekst
\paragraph{odlomak}
neki tekst
\subparagraph{pododlomak}
neki tekst
%-------prva sekcija i njene podsekcije kraj----------------

%--------druga sekcija i njene podsekcije--------------------
\section{sekcija2}
neki tekst
\subsection{podsekcija2}
neki tekst
\subsubsection{podpodsekcija2}
neki tekst
\paragraph{odlomak}
neki tekst
\subparagraph{pododlomak}
neki tekst
%-------druga sekcija i njene podsekcije kraj----------------

%i sad dalje sekcija3 i njene podsekcije, pa sekcija4 i tako dalje po potrebi


%-----------------------------------------------------------------------------
%-------------bibliografija---------------------------------------------------

\begin{thebibliography}{99} %ako je manje od 10 referenci stavimo 9 umjesto 99
\bibitem{oznaka1} referenca1
\bibitem{oznaka2} referenca2
\end{thebibliography}

%-------------bibliografija kraj-----------------------------------------------
%------------------------------------------------------------------------------

\end{document}
