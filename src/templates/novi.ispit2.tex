\documentclass[a4paper,11pt]{article}
\usepackage[croatian]{babel}
\usepackage{amsfonts} \usepackage{amssymb}
\usepackage{amsmath} \usepackage{fancybox}
\addtolength{\voffset}{-2cm} \addtolength{\textheight}{3cm}
\addtolength{\textwidth}{2cm} \addtolength{\hoffset}{-1cm}
\pagestyle{empty}

\begin{document}
\newcounter{grupe} %ne brisati!!!

\begin{center}
\shadowbox{Naziv kolegija}\\[5pt]
datum\\[5pt]
-- pismeni ispit --
\end{center}

\vspace{0.1cm}

\paragraph{Napomena.} Za prolaznu ocjenu, iz svake grupe treba rije\v{s}iti najmanje jedan zadatak
u potpunosti i ukupni broj bodova treba biti ve\'{c}i od 50\%.\\[0.8cm]

\noindent\shadowbox{I. grupa}

\begin{enumerate}
\item prvi zadatak
\item drugi zadatak
\item tre\'ci zadatak
\begin{enumerate}
\item a) dio tre\'ceg zadatka
\item b) dio tre\'ceg zadatka
\end{enumerate}

\setcounter{grupe}{\value{enumi}} %ne brisati!!!
\end{enumerate}

\vspace{1cm}

\noindent\shadowbox{II. grupa}

\begin{enumerate}
\addtocounter{enumi}{\value{grupe}} %ne brisati!!!

\item \v{c}etvrti zadatak
\item peti zadatak
\item \v{s}esti zadatak

\setcounter{grupe}{\value{enumi}} % izbrisati ako nema trece grupe zadataka
\end{enumerate}

\vspace{1cm}

\noindent\shadowbox{III. grupa}

\begin{enumerate}
\addtocounter{enumi}{\value{grupe}}
\item sedmi zadatak
\item osmi zadatak
\end{enumerate}

\end{document}
