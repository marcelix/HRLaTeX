% dodatak.tex
\chapter{Dodatak}



% \section*{Dokaz neadekvatnosti  Kneale izvoda}
% 
% \newpage




\section{Tekst prijave teme (OBAD)}


\subsection*{Sažetak HR}
% 
Metoda semantičkih stabala je potpuni postupak za logiku prvog reda. Okretanje semantičkih stabala naopako daje
Gentzenov sustav multiplarnih sekventi bez reza. Ovaj dobro poznati rezultat pokazuje kako je Gentzenov sustav
multiplarnih sekventi ustvari notacijska varijacija Bethove metode semantičkih stabala.

Mi ćemo pokazati kako se sustav multiplarnih prirodnih dedukcija (prirodnih dedukcija s multiplarnom relacijom
posljedice) može povezati s Bethovim semantičkim stablima. Pokazati ćemo da i za multiplarni sustav prirodnih dedukcija
vrijedi algoritamska ekvivalentnost s metodom semantičkih stabala.


\subsection*{Abstract EN}
% 
\textit{The method of Beth semantic trees is a complete procedure in logic. Semantic trees turned upside down give
Gentzen’s cut-free multiple sequent proofs. This is a well known fact which reveals that we can regard Gentzen’s
cut-free multiple sequents system as a notational variation of Beth’s semantic tree method.}

\textit{%
We would like to prove that multiple conclusion natural deduction systems can, similarly to sequent systems, be
associated with Beth’s semantic trees. We will show there is an algorithmic equivalnce between multiple conclusion
natural deduction sytems and Beth's semantic tree method.}



\subsection*{Uvod i pregled dosadašnjih istraživanja}
% 
Metoda semantičkih stabala (tabloa) razvijala se potpuno nezavisno od Gentzenovih sustava prirodne dedukcije. Bethova
semantička stabla proizlaze iz semantičkog pristupa logici i orjentirana su na klasičnu logiku i njezinu semantiku.

Prirodne dedukcije, s druge strane, proizlaze iz sintaktičkog pristupa logici - iz izgradnje dokaza i pravila
zaključivanja. U tom su smislu prirodne dedukcije prikladne za minimalnu i intuicionističku logiku iz kojih se klasična
logika dobije tek dodatkom pravila "tertium non datur" (TND) ili pravila "reductio ad absurdum" (RAA).

Mi ćemo pokazati kako se sustav multiplarnih prirodnih dedukcija (prirodnih dedukcija s multiplarnom relacijom
posljedice) može povezati s Bethovim semantičkim stablima. Pokazati ćemo da i za multiplarni sustav prirodnih dedukcija
vrijedi algoritamska ekvivalentnost s metodom semantičkih stabala. Na ovaj način pokazali bi semantički pristup
sustavima multiplarnih prirodnih dedukcija.

Posljedica ovoga je potpunost našeg sustava multiplarnih prirodnih dedukcija (za razliku od Kneale sustava). Za razliku
od Shoesmith i Smileyevog pristupa naš sustav multiplarnih prirodnih dedukcija je dobro motiviran i jednostavan.



\subsection*{Cilj i hipoteze istraživanja}
% 
Hipoteza istraživanja je da postoji algoritam koji pokazuje algoritamsku ekvivalenciju multiplarnih prirodnih dedukcija
(sintaktičkog pristupa) i Bethovih semantičkih stabala.
Cilj istraživanja je potvrditi hipotezu izradom korektnog algoritma.


\newpage

\section{Skica predgovora}
\vspace*{10cm}


U nekim dijelovima izostavljeni su neki dijelovi standardnih definicija koji se odnose isključivo na intuicionističku logiku.

 \bigskip
 \centerline{***}
 \bigskip



\newpage
\section{Post Scriptum}

\subsection*{Prednosti}
\begin{enumerate}
 \item Nema hipotetskih pravila: Sva pravila zaključivanja su lokalna; \newline
       pravila izvoda su REZ/SPOJ i  KONTRAKCIJA/DUPLIKACIJA
 \item Direktna analitičnost
 \item Semantički postupak, direktno prepisivanje u tabloe i sekvente
 \item "Prirodnost" očuvana: pravila introdukcije i eliminacije su ili jednaka ili "prirodnija"
\end{enumerate}


\subsection*{Nedostaci}
\begin{enumerate}
 \item Razilaženje s intuicionističkom logikom (nema više "2 za 1") 
 \item Razilaženje sa "prirodnošću" (prema Gentzenu) singularne konkluzije
\end{enumerate}





% +++++++++++++++++++++++++++++++++++++++++++++++++++++++++++++++++++=
% 
\chapter{Razvrstati}

\section{Singularna konkluzija u LJ}
\begin{itemize}
 \item Gentzenov račun sekventi LJ za intuicionističku logiku ima restrikciju singularne konzekvente -- ograničavanjem
sekventi na 
       točno jednu premisu dobije se račun intuicionističke logike. 
       Ovo ograničenje nije nužno u svim pravilima. Maheara\index{Maheara} je 1954.~predstavio račun LJ' (vidi
\cite{negri2008structural}) u kojem 
       restrikcija singularne konkluzije vrijedi samo  za sekventna pravila \rulem{|- -} i  \rulem{|-\to}.
 
	Ostala pravila su iz računa LK klasične logike (vidi str.~\pageref{gentzen:LK}).

\item \textit{Što to znači za multiplarne prirodne dedukcije za intuicionističku logiku?}\newline
       U MPD računu moramo najprije izbaciti aksiom \rulen{TND}. Od ostalih "novih" multiplarnih pravila nužno je
izbaciti \intr{\to}
% 
\begin{mathpar}
 \inferrule{\ }{A \\ A\to B}
\end{mathpar}
(ako stavimo $B \equiv \neg A$ ponovno dobijemo \rulen{TND}).
% 
\end{itemize}


\begin{center}
 \includegraphics[width=\textwidth]{LJm-manu.pdf}
\end{center}


\sik{Propozicijska logika ili logika prvog reda? -- odluci se, jer inace fali jos jedno restr. pravilo za LJ'}





\section{O analitičkim logičkim računima}
% \sik{Vidi Gentzen - pronadji, Smullyan - pocetak}
% \sik{Pojam analitckog racuna uveo je Smullyan u cite-FOL. Ideja je... }

Za logički račun kažemo da je \textbf{analitički}\index{analitički!račun} ako u njemu vrijedi svojstvo 
potformulnosti\index{potformulnost} 
--- 
svaki dokaz/izvod/dedukcija može se prevesti u izvod u kojem su sve formule potformule  premisa i konkluzija.

Račun semantičkih tabloa je trivijalno analitički račun. Sekventni račun LK je analitički (naopaki tabloi, Hauptsatz).
Smullyan svoju verziju semantičkih stabala zove analitička semantička stabla.

Prirodne dedukcije nisu   \textit{a priori} analitičke. Teoremi o normalnoj formi daju
egzistenciju analitičke dedukcije, ali postupak nije jednostavan -- direktan (referenca, vidi \ref{})
\fxnote{ovo ne razumijem, kako RAA ulazi u to...}




% \begin{center}
% %  \includegraphics[width=\textwidth]{Image-0252.png}
% \end{center}


Vidi\newline
{\smaller
\url{http://en.wikipedia.org/wiki/Analytic_proof}
}

\sik{pojam analiticki racun -- prvi spominje Smullyan. Nema ne-kontroverzne definicije, ali 
za prirodne dedukcije i sekvente je nesporno\newline 
A) analiticki izvod prirodnih dedukcija je...\newline
B) u racunu sekventi analiticki izvodu su izvodi bez reza.
}





\newpage
\section{Razgranati Hilbertovi dokazi}
\begin{defn}[Razgranati Hilbertov dokaz]
 TODO. Vidi \cite{sika-disertacija}.
\end{defn}

\begin{pr}[Prepisivanje Kneale dedukcije u razgranati Hilbertov dokaz]\strut\newline 
\begin{center}
 \includegraphics[width=12cm]{pic/dropbox-pics/Image-0298.png}
\end{center}
\end{pr}


\subsubsection*{Algoritam prepisivanja....}
\begin{verbatim}
 1: kreni iz neke premise
 2: kotrljaj nizbrdo (fork je ok, p
\end{verbatim}


% \vfill

\newpage
\begin{pr}[A što je s ciklusima (u multiplarnim dedukcijama)?]\strut\newline 
\begin{center}
 \includegraphics[width=14cm]{pic/dropbox-pics/Image-0296.png}
\end{center}
\end{pr}



\newpage
\section{Parsing trees i semantičko stablo}
\centerline{"parse tree" $\mapsto$ stablo raščlambe}


\vspace*{3cm}



\newpage

 
\begin{longsik}

\section*{Ilustracija Prawitzovog principa inverzije}
 \subsection*{Introdukcije definiraju značenje veznika}
\subsubsection*{Multiplarna ilustracija principa inverzije}
% 

Neka je $A'\in \{A, -A\}$ i $B'\in \{B, -B\}$.
Neka su formule $A'$ i $B'$ nužne i dovoljne za uvođenje veznika $A\circ B$. 
Tada imamo sljedeću introdukciju:
\begin{mathpar}
   \inferrule*[Left={\intr{\circ}}]{A' \\ B'}{A\circ B}
\end{mathpar}
Iz $A\circ B$ se $\circ$ može eliminirati  i zaključiti  $A'$ ili $B'$:
% 
\begin{mathpar}
 \inferrule*[Left={\elim{\circ}}]{A\circ B}{A'} \and \inferrule*[Left={\elim{\circ }}]{A\circ B}{B'}
\end{mathpar}
Pripadna eliminacija je
 


Umjesto $\circ$ mogli smo ubaciti bilo koji veznik $\wedge, \vee$ i $\to$.


Slično, ako je $A'$ dovoljna za $A\circ B$ i ako je $B'$ dovoljna za zaključiti $A\circ B$.


Po kontrapoziciji 
Ako je za zaključiti $A\circ B$ dovoljna formula $A'$ ili formula $B'$, pripadna 
introdukcija veznika $\circ$ glasi
\begin{mathpar}
 \inferrule*[Left={\intr{\circ}}]{A'}{A\circ B}
 \and 
 \inferrule*[Left={\intr{\circ}}]{B'}{A\circ B}
\end{mathpar}
% 
Pripadna eliminacija je:
% 
\begin{mathpar}
   \inferrule*[Left={\elim{\circ}}]{A\circ B}{A' \\ B'}
\end{mathpar}

Pravila za kondicional $\to$ možemo prilagoditi da se uklope u ovu priču.
% 
\begin{mathpar}
 \inferrule{-A}{A\to B} \and \inferrule{B}{A\to B} \and \inferrule{A\to B}{-A \\ B}\ .
\end{mathpar}

Svaki od ovih veznika dovoljno je definirati preko introdukcije ili eliminacije. 
Drugo skup pravila je posljedica definicije.

\begin{align*}
  A \circ B & \iff A' \wedge B' \Rightarrow A' \\
  A \circ B & \iff A' \wedge B'   \Rightarrow B'\\
% \intertext{ slijedi }
%  \inferrule*{ A' \\ B' }{A \circ B} \qquad  
\end{align*}
\begin{mathpar}
  \inferrule*{ A' \\ B' }{A \circ B} \and \inferrule*{A\circ B}{A'} \and \inferrule*{A\circ B}{B'}
\end{mathpar}


\begin{align*}
  A \circ B & \iff A' \vee B' \Leftarrow A' \\
  A \circ B & \iff A' \vee B' \Leftarrow B'\\
% \intertext{ slijedi }
%  \inferrule*{ A' \\ B' }{A \circ B} \qquad  
\end{align*}
\begin{mathpar}
  \inferrule*{A \circ B}{ A' \\ B' } \and \inferrule*{A'} {A\circ B}\and \inferrule*{B'}{A\circ B}
\end{mathpar}




\end{longsik}





\newpage
\begin{longsik}

\subsection*{\textit{Ex falso sequitur quodlibet} -- što s $\pmb{\top}$ i $\pmb{\bot}$?}

Među multiplarnim pravilima nismo eksplicitno naveli pravila za $\top$ i $\bot$ --
eliminaciju \elim{\bot} (\textit{ex falso quodlibet}), introdukciju \intr{\top} i \intr{\bot}:
% 
\begin{mathpar}
\inferrule*{\bot}{\quad * \quad }
  \and 
\inferrule*{\quad * \quad}{\top}
%   \and 
% (\textit{ex falso quodlibet})
\end{mathpar}
% 
ili originalne Kneale verzije
% 
\begin{mathpar}
 \inferrule*{ A \\ -A}{*}
 \and 
 \inferrule*{ * }{A \\ A\to B}
 \and 
 \inferrule*{ * }{A \\ -A}
 \ .
\end{mathpar}
% 
pri čemu se na mjestu simbola "$*$" može pojaviti proizvoljna formula, ili čak prazna premisa odnosno konkluzija (kao u
našoj verziji).

Za našu verziju multiplarnih pravila, s praznim premisama i praznim konkluzijama vrijedi potformulnost, što se pokazalo
dobrim u 
traženju analitičkih dedukcija.

\bigskip

Uočimo da je u definiciju dedukcija ugrađeno da je $\Pi$ dedukcija od $\Delta$ iz $\Gamma$ 
ako je skup premisa podskup od $\Gamma$ i skup konkluzija podskup od $\Delta$.

Tako je dedukcija
\begin{mathpar}
   \inferrule*{ A \\ -A}{  }
\end{mathpar}
s praznom konkluzijom dedukcija od $A, -A |- C$.
% slabljenje nije potrebno.

\bigskip 

Što ako u formulama $\Gamma$ i $\Delta$ ima konstanti $\bot, \top$? 
Možemo ih pokušati eliminirati primjenama pravila apsorpcije 
($A\vee\bot \equiv A$, $A\vee \top \equiv \top$ i sl.). 
% 

Ako neka od konstanti ostane kao formula sekvente, imamo sljedeće slučajeve:
% Apsorpcija na sekventama:
% 
\begin{mathpar}
%   \text{\textup{(a)}} \ 
  \inferrule{\Gamma, \top |- \Delta}{\Gamma |- \Delta} 
  \and
  \inferrule{\Gamma |- \Delta, \bot}{\Gamma |- \Delta}
  \and
  \inferrule{\Gamma, \bot |- \Delta}{ \bot |- \Delta}
  \and
  \inferrule{\Gamma |- \top, \Delta}{ \Gamma |- \top }
\end{mathpar}



    % 
    U prva dva slučaja $\top$ i $\bot$ ignoriramo.
    Treći i četvrti slučaj su trivijalni.
    % 

    Ostaje pitanje kako napraviti dedukciju od $\bot |- A$? 
    Mogu li u tom slučaju $\bot$ zamijeniti s $A\wedge -A$?, i $\top$ s $A\vee -A$?


    Dodati pravila?
    \begin{mathpar}
      \inferrule*[Left={\elim{\bot}}]{\quad \bot\quad }{\phantom{\bot} } \and \inferrule*[Left={\intr{\top}}]{ }{\quad
\top\quad }
    \end{mathpar}
    % 
\end{longsik}



\begin{longsik}
Ekvivalentna disjunktivna forma je
\[
 AB \vee ABC \vee A\overline{C}B = AB(\top + C +\overline{C}) = AB
\]


U nastavku ćemo slobodno koristiti klauze u formulama umjesto odgovarajućih disjunkcija njihovih elemenata.
Na primjer, pisati ćemo
\[
 \{A, -B\} \wedge \{C, D\}, \text{ ili } \mathcal{L}_1 \wedge \mathcal{L}_2
\]


\end{longsik}









\newpage





\subsection*{Priča o konkatenaciji stabala}
 
\begin{defn}[Konkatenacija stabala]
Neka su $\tau_1$ i $\tau_2$ korjenska stabla. Stablo koje nastaje dopisivanjem po jedne kopije stabla $\tau_2$ na svaki
list stabla $\tau_1$ zovemo \textbf{konkatenacija stabala}\index{konkatenacija!stabala} $\tau_1$ i $\tau_2$.

Konkatenacija semantičkih stabala je semantičko stablo:
\begin{itemize}
 \item $\tau_1$ i $\tau_2$ se spajaju samo na otvorenim granama od $\tau_1$
 \item Na dobivenom stablu zatvorimo grane koje se mogu zatvoriti
\end{itemize}

\end{defn}


\begin{center}
 \input{pic/dropbox-pics/phd_konjunkcija.pgf}
\end{center}


\sik{-- begin --}
% =================================================
\begin{pr}[Primjer konkatenacije stabala]\strut \ 

\sik{nije dobro, fale zahtjevi... ovo je sazetak koji jos nisam definirao}
\begin{center}
\input{pic/phd_konjunkcija_stabala.pgf}
\end{center}

\end{pr}  
\sik{a kaj je s korijenima? Ova stabla nemaju korijene... a neka bi mogla...}
% =================================================
\sik{-- end --}
 
\begin{defn}[Konkatenacija semantičkih stabala]\index{konkatenacija!semantičkih stabala}%
 Semantičko stablo koje nastaje konkatenacijom otvorenih dijelova semantičkih stabala $\tau_1$ i $\tau_2$, te zatvaranjem
grana koje se mogu zatvoriti zovemo \textbf{konkatenacija semantičkih stabala} $\tau_1$ i $\tau_2$.
\end{defn}

Uobičajenu izgradnju Bethovog semantičkog stabla možemo interpretirati kao niz konkatenacija stabla
kojeg gradimo i (malih) stabala zahtjeva.

%  
%  \bigskip
%  \centerline{***}
%  \bigskip
 
 
U nastavku će nam biti posebno zanimljiva semantička stabla konjunktivnih formi. 




\section{Konjunkcija semantičkog sekventnog stabla}


% \newpage
% 
% \sik{Ovo u neki dodatak...}


Sekventna semantička stabla također su \textit{semantička} stabla (kao i Bethova): na njima  smo definirali pojmove
završene, zatvorene i otvorene grane. Zato možemo definirati i 
konjunkciju sekventnih semantičkih stabala.

\begin{defn}[Konjunkcija sekventnih semantičkih stabala]
Neka su $S_1$ i $S_2$ semantička sekventna stabla
\begin{itemize}
 \item Sve sekvente od $S_1$ dopuniti: u premisu (konkluziju) prepisati premisu (konkluziju) od korijena od $S_2$.
 \item U svaku dopisanu kopiju od $S_2$ treba u sve sekvente dopuniti: u premise (konkluzije) dopisati premise
(konkluzije) iz spojnog lista
\end{itemize}
\end{defn}
\fxnote{napraviti konjunkciju sekventnih sem. stabala}


\subsection*{Slabljenje za $\nmodels$}

\sik{ 
O tome kad budem pisao o prepisivanju sparivanja u sekventni dokaz...
}


\sik{%
Račun multiplarnih dedukcija je adekvatan za Knealejev sustav pravila \newline
(tj. za relaciju logičke posljedice koju ona induciraju)
\newline
(ref: Sikin PhD)
}