%  PREAMBULA za PhD rad.
% 
% 
% defs....
% preambula.tex

% BEGIN ucitavanje paketa
\usepackage{etex} % nuzno, jer trosim puno paketa.


\usepackage{xparse}  % za bolji newcommand


\usepackage{xcolor}
\usepackage{graphicx}


\usepackage{array}



% BEGIN TikZ
\usepackage{tikz}
\usetikzlibrary{%
    trees,%
    calc,%
    shapes.symbols,%
    shapes.arrows,%
    chains,%
    snakes,shapes,%
}

\usepackage{tikz-qtree}% za tabloe.
\usepackage{forest}% za gusta,siroka stabla
% END TikZ




% Indeks
\usepackage{makeidx}
\usepackage[nottoc,numbib]{tocbibind}



\usepackage{pdfpages}





% BEGIN Matematika (AMS,...)

\usepackage{amsmath}
\usepackage{amssymb}
\usepackage{amsthm}
\usepackage[thicklines]{cancel}
\usepackage{mathtools}
\usepackage{dashrule}


\usepackage[a4paper]{geometry}

% END Matematika

\usepackage[croatian,%
% draft,
noinline,%
% silent,%
footnote,%
% final,%
]{fixme}


% BEGIN bloody FLOAT sekcija

% \usepackage{float}
\usepackage{newfloat}
\usepackage{subfig}
\usepackage{caption}
\usepackage{floatrow}
\usepackage{placeins} %FloatBarrier?



\DeclareFloatingEnvironment[name={},placement=H]{fakefloat}


\DeclareCaptionFormat{fakefloat}{format=hang,justification=raggedright}
\captionsetup[figure]{
  format=hang,
  labelfont=bf,
  textfont={small},
  justification=justified,
%   width=.75\textwidth,
}

\captionsetup[table]{
  format=hang,
  labelfont=bf,
  textfont={small},
  justification=justified,
%   width=.75\textwidth,
}

\usepackage[algoruled,vlined,algochapter]{algorithm2e}
\renewcommand{\algorithmcfname}{Algoritam}%

% END FLOAT sekcija


% \usepackage{caption}

\usepackage{pgfplots} % ? mislim da ne treba

% BEGIN Fontovi

\usepackage{times}
\usepackage{relsize}
\usepackage{microtype}
\usepackage{turnstile}
\usepackage{centernot}
\usepackage{manfnt}
\usepackage{bbding}
% 

% -------------  FONT => PALATINO 
\usepackage{tgpagella}
\usepackage{mathpazo}
% \onehalfspacing %   <---- 1.5 prored...  ONE-HALF-SPACING
% 
% \usepackage[scaled]{beramono}
\usepackage[scaled]{inconsolata} % TT fixed font
%---------------------------------

% END Fontovi






%BEGIN  za typesetting logickih stvari

% \usepackage{centernot}
%END Logika





% BEGIN Liste, Okviri i sl.

\usepackage{framed}

\usepackage{paralist}
\usepackage{enumerate}
\usepackage[shortlabels]{enumitem} %  

% END Liste, Okviri i sl.


\usepackage{url}



% END ucitavanja paketa



% NIJE PODESENO 
\geometry{
        %       papersize={168mm, 240mm},
        papersize={210mm,297mm}, % ovo je A4
        %        
        inner=2.5cm, 
        outer=2.50cm,
        top=25mm,
        bottom=25mm, % povecao donju marginu za 3mm...
        bindingoffset=5mm,
        verbose,
        includeheadfoot,
%         dvips=true,
        %       mag=1120,       
}

% \geometry{
%         %       papersize={168mm, 240mm},
%         papersize={210mm,297mm}, % ovo je A4
%         %        
%         inner=2.91cm, 
%         outer=2.00cm,
%         top=21mm,
%         bottom=22mm, % povecao donju marginu za 3mm...
%         bindingoffset=1mm,
%         verbose,
%         includeheadfoot,
% %         dvips=true,
%         %       mag=1120,       
% }

% conf_operatori.tex

\DeclareMathOperator{\modulo}{mod}
\DeclareMathOperator{\rad}{rad}


% \DeclareMathOperator{\Arsh}{Ar\,sh}
% \DeclareMathOperator{\sh}{sh}


\newcommand{\R}{\mathbb{R}}
\newcommand{\Q}{\mathbb{Q}}
\newcommand{\Z}{\mathbb{Z}}
\newcommand{\N}{\mathbb{N}}
\newcommand{\C}{\mathbb{C}}
\newcommand{\D}{\displaystyle}
\newcommand{\F}{\mathbf{F}}

\newcommand{\M}{\mathcal{M}}
\newcommand{\UU}{\ensuremath{U}}

\newcommand{\U}{U}

% \DeclareMathOperator{\ctg}{ctg} \DeclareMathOperator{\tg}{tg}
% \DeclareMathOperator{\arcctg}{arcctg}
% \DeclareMathOperator{\arctg}{arctg}
% \DeclareMathOperator{\sign}{sign} \DeclareMathOperator{\ch}{ch}


\newcommand{\kond}{\rightarrow}
\newcommand{\bi}{\leftrightarrow}
\newcommand{\impl}{\Rightarrow}
\newcommand{\nimpl}{\nRightarrow}
\newcommand{\ekvi}{\Leftrightarrow}
\newcommand{\nekvi}{\nLeftrightarrow}
\newcommand{\non}[1]{\overline{#1}}
\newcommand{\modelss}{\models _s}

\newcommand{\forallm}[1]{\forall #1 \,} 
\newcommand{\existsm}[1]{\exists #1 \,}
% \def{\iff}{\ekvi}
\let\iff\ekvi
\let\bkond\leftrightarrow


\newcommand*{\nDiamond}{%
\mathrel{%
\vcenter{\offinterlineskip%
\hbox{\ensuremath{\Diamond}}\vskip-1.6ex\hbox{\ensuremath{\diagup}}
}}}
% \newcommand*{\nDiamond}{\not\mathrel{\Diamond}}
% {\Diamond\!\!\!\!\raisebox{1pt}{\ensuremath{\diagup}}}}}
% \newcommand{\nBox}{\ensuremath{\not\!\!\Box}}
% \newcommand{\nBox}{\ensuremath{\not\mathrel{\Box}}}
\newcommand{\nBox}{\ensuremath{\centernot\Box}}

\newcommand{\mybox}{\Box}
\newcommand{\mydi}{\Diamond}

\newcommand{\nbox}{\nBox} % a sta rec, brkam nazive
\newcommand{\ndi}{\nDiamond}

\newcommand{\nmodels}{\mathrel{\not\! |=}}

\newcommand{\rulef}{\upshape\sffamily}
\newcommand{\rulen}[1]{{\smaller\rulef{(#1)}}}
\newcommand{\srulen}[1]{\footnotesize (\rulef{#1})}
\newcommand{\rulep}[1]{(#1)}
\newcommand{\rulem}[1]{($ #1$)}
% 
% \newcommand{\elim}[1]{{\text{\upshape\smaller({\rulef E}} {\ensuremath{#1}}}\text{\upshape\smaller)}}

\newcommand{\elim}[1]{{  \text{\smaller\rulef (E{\ensuremath{#1}})}}}
\newcommand{\intr}[1]{{  \text{\smaller\rulef (I{\ensuremath{#1}})}}}
% \newcommand{\intr}[1]{{\text{\upshape\smaller({\rulef I}} {\ensuremath{#1}}}\text{\upshape\smaller)}}
% 

% \newcommand{\andown}[1]{{  \text{  {\ensuremath{  \smash      { \boxed{#1} \atop {\larger[4] \downarrow }  }  }}}  }}
% \newcommand{\anup}[1]{{    \text{  {\ensuremath{  \smash       {{{\larger[4] \uparrow} \atop \boxed{#1}}  } }}}  }}
% 
\newcommand{\anup}[1]{{%
    \text{ \smaller {\ensuremath{  \smash{\begin{array}{c} {\uparrow}\\ \boxed{#1} \\ \vphantom{\uparrow} \end{array}} }}}  }}

\newcommand{\andown}[1]{{%
    \text{ \smaller {\ensuremath{  \smash{\begin{array}{c} \vphantom{\uparrow}\\ \boxed{#1} \\ {\downarrow} \end{array}} }}}  }}


\newcommand{\gentzi}[1]{$(|- #1)$}
\newcommand{\gentze}[1]{$(#1 |-)$}


\newcommand{\ext}{\mathsf{ext}}
\newcommand{\inte}{\mathsf{int}}
\newcommand{\Atom}{\mathcal{A}t}



\newcommand{\odbaci}[1]{\footnotesize{(#1)}}


\newcommand{\subs}[2]{\ensuremath(#1 / #2 )}


\newlength{\wedgelen}
\settowidth{\wedgelen}{$\wedge\,\,$}

\newcommand{\wedgeveekond}{
% \ensuremath{
% 
\mathop{
\makebox[\wedgelen]{
\ensuremath{
\begin{array}{c}
 \strut \\
 \strut \\
 \wedge \\
 \vee   \\
 \kond
\end{array}}}}\,\,}
% }


%  FIX/HACK -- Sikic zeli produljiti |-  u |--
% \mathlig{|-}{\mathop{\vdash\!\!\!\!{-}}}

% Jednostavna izvodivost s multiplarnim izvodima
% SK -- Standardne Kneale dedukcije
\def\skder{|-_{_{\!\!\!\tiny \textsf{\textup{KN}}}}}

% Multiplarna izvodivost (s jednostavnim ciklusima)
% MD -- Multiplarne Dedukcije
\def\mdder{|-_{_{\!\!\!\tiny \textsf{\textup{MD}}}}}


\newcommand{\marcfrac}[2]{\ \begin{array}{c}  #1 \\ \hline\hline #2 \end{array}}
\newcommand{\qfrac}[2]{\ \begin{array}{c}  #1 \\ \hline #2 \end{array}}
\newcommand{\nfrac}[2]{\ \begin{array}{c}  #1 \\  #2 \end{array}}
% \newcommand{\nfrac}[2]{   {#1 \atop  #2 }}


\newlength{\foralllen}
\settowidth{\foralllen}{$\forall\,$}

\newcommand{\forallexists}{
\mathop{
\makebox[\foralllen]{
\ensuremath{
\smash{
\begin{array}{c}
 \strut \\
%  \strut \\
 \forall \\
 \exists   
% \\
%  \kond
\end{array}}}}}}

\newcommand\bsrule[1]{%
\boxed{#1}\quad%
}




% BEGIN Hacks
\newcommand{\zbox}[1]{\makebox[0cm][l]{#1}}
\newcommand{\zcbox}[1]{\makebox[0cm][c]{#1}}
\newcommand{\zrbox}[1]{\makebox[0cm][r]{#1}}








\newcommand{\sik}[1]{%
% 
\protect\opt{sika-komentari}{%
\par 
\begingroup
\RaggedRight
% \marginpar[$\Rightarrow$]{\XSolidBold }%
% \begin{snugshade*}%
\textcolor{violet}{
% \begin{leftbar}%
% 
\zrbox{\smaller \XSolidBold \hspace*{0.5cm}\strut}%
\textcolor{violet}{\small \texttt{#1}}%
% \par\relax
% \end{leftbar}%
}%
% \end{snugshade*}%
\endgroup
\par 
}
}

\newenvironment{longsik}{%
\sik{\centerline{-------- BEGIN -------}}
\color{red}\begingroup \smaller 
\begin{quote}%
% \begin{leftbar}
% end color
}
{
% \end{leftbar}%
\end{quote}
\endgroup
\sik{\centerline{-------\ \ END\ \   -------}}
}


\newcommand{\CC}[1]{%%
\tikz[baseline=(o.base),inner sep=1pt, minimum size=0pt] \node (o) [draw,circle]{\ensuremath{#1}};%
}

\newcommand{\rREC}[1]{%
\tikz[baseline=(o.base),inner sep=3pt, minimum size=0pt] \node (o) [draw,rounded corners=2pt]{\ensuremath{#1}};
}
\providecommand{\non}[1]{\overline{#1}}

\newcommand{\semT}[1]{%
\ensuremath{(#1\top)}%
}
% 
\newcommand{\semF}[1]{%
\ensuremath{(#1\bot)}%
}

% END Hacks




\graphicspath{{pic/}{pic/dropbox-pics}}



% ======================================

% BEGIN teoremi

\theoremstyle{plain}


\newtheorem{teorem}{Teorem}[chapter]
\newtheorem*{hauptsatz}{Hauptsatz}
\newtheorem{kor}[teorem]{Korolar}
\newtheorem{lema}[teorem]{Lema}


\newtheorem{prop}[teorem]{Propozicija}
% \newtheorem{kor}{Korolar}[chapter]
\newtheorem{tm}[teorem]{Teorem}
\newtheorem{pr}[teorem]{Primjer}
\newtheorem{korolar}[teorem]{Korolar}
\newtheorem*{princip:inverzije}{Princip inverzije}
\newtheorem*{hintikkina:lema}{Hintikkina lema}

% \theoremstyle{remark}
\newtheorem*{notabene}{N.B}
\newtheorem{napomena}[teorem]{Napomena}




\newtheoremstyle{mydef}% name of the style to be used
{2\topsep}% measure of space to leave above the theorem. E.g.: 3pt
{2\topsep}% measure of space to leave below the theorem. E.g.: 3pt
{\itshape}% name of font to use in the body of the theorem
{}% measure of space to indent
{\bfseries}% name of head font
{.}% punctuation between head and body
{\newline}% space after theorem head; " " = normal interword space
{}% Manually specify head

\theoremstyle{mydef}
\newtheorem{defn}[teorem]{Definicija}


% END teoremi


% ======================================



\hypersetup{
  pdftitle={Multiplarne prirodne dedukcije, PhD Thesis},
  pdfauthor={Marcel Maretic},
  pdfsubject={Mathematics},
  pdfcreator={pdfLaTeX},
}
\hypersetup{colorlinks=true, linkcolor=blue}
% \usepackage[backend=biber]{biblatex}


%BEGIN biblatex, umjesto bibtex
\usepackage[%
sortcites=true,
% style=numeric,
style=alphabetic,
% style=authoryear,
clearlang=true,
% babel=none,
backend=biber,
backrefsetstyle=setandmem,
]{biblatex}
\addbibresource{reference.bib}

% % % % \DeclareFieldFormat{formaturl}{\newline #1}
% % % % \newbibmacro*{url+urldate}{%
% % % % \printtext[formaturl]{%
% % % %   \printfield{url}}%
% % % %   \iffieldundef{urlyear}
% % % %     {}
% % % %     {\setunit*{\addspace}%
% % % %      \printtext[urldate]{\printurldate}}}
     
     
%      ZA NEWLINE prije URL-a u BIBLIOGRAFIJI
\DeclareFieldFormat{url}{\newline\mkbibacro{URL}\addcolon\space\url{#1}}
\newbibmacro*{url+urldate}{%
 \printfield[url]{url}%   
 \iffieldundef{urlyear}     
 {}     
 {\setunit*{\addspace}%      
 \printtext[urldate]{\printurldate}}}


\DefineBibliographyStrings{english}{%
editor={Urednik},
in = {},
phdthesis = {Doktorska disertacija},
and={i},
}



\definecolor{shadecolor}{RGB}{250,250,255}%,240,240}

\ExecuteBibliographyOptions{
firstinits=true,
% url=false,
}

\AtBeginBibliography{\def\UrlFont{\small\tt}}

 

%END biblatex



%opening







% \write18{sudo apt-get update}

% \makeindex
% 
\newcommand*{\lettergroup}[1]{%
{\hfil\sffamily\bfseries {\large #1} \hfil\hfil\hfil%
% \strut\\[-0.25ex]
}}



\newcommand{\sectionBreak}{%
\centerline{$\star \ \star \ \star$}%
}




% >>>>>>>>>>>>>>>>>>>>>>>>>>>>>>>>>>>>>>>>>>>>
%  TABLO makroi za TikZ



% \newcommand{\TT}[1]{%
% \node at ($(#1.east) + (0.8,0)$) {$\top$};
% }
\newcommand{\mTT}[2]{%
\node at ($(#1.east) + (0.5,0)$) {$\top$};
}

% \newcommand{\CTT}[1]{%
% \node[circle,draw,inner sep=1.3pt] at ($(#1.east) + (0.8,0)$) {$\top$};
% }

\newcommand{\rTT}[2]{%
\node[anchor=west] at ($(#1.east) + (0.8,0)$) {#2};
}


\DeclareDocumentCommand{\TT}{O{0.8} m}{%#
\node at ($(#2.east) + (#1,0)$) {$\top$};
}

\DeclareDocumentCommand{\CTT}{O{0.8} m}{%#
\node[circle,draw,inner sep=1.3pt] at ($(#2.east) + (#1,0)$) {$\top$};
}


\DeclareDocumentCommand{\FF}{O{0.8} m}{%#
\node at ($(#2.east) + (#1,0)$) {$\bot$};
}

\DeclareDocumentCommand{\CFF}{O{0.8} m}{%#
\node[circle,draw,inner sep=1.3pt] at ($(#2.east) + (#1,0)$) {$\bot$};
}

 




\newcommand{\drawFork}[3]{%
% #1 je branch, #2 i #3 su donji cvorovi...
% 
    %     Odredi vertikalnu sredinu
    \coordinate (cy) at ($(#1.south)!0.5!(#2.north)$);

    %         Pomakni vertikalnu sredinu ispod (branch) cvora.
    \coordinate (cr) at (cy -| #1.south);
    
    \draw [  thick, shorten >=2pt ] (cr) -- (#1.south)
           [ thick, shorten >=2pt ] (cr) -| (#2.north)   
           [ thick, shorten >=2pt ] (cr) -| (#3.north)
}

% % % % % % \DeclareDocumentCommand{\potcrtajCvor}{O{0.4} m}{%#
% % % % % % % Potcrtavanje cvorova (zatvaranje grana tabloa)
% % % % % % % \newcommand{\potcrtajCvor}[1]{%
% % % % % % \draw[very thick] (#2.south west) +(-#1,0) -- ($(#2.south east) + (#1,0)$);
% % % % % % % let p1 = (#1.south - #1.east)
% % % % % % }

\DeclareDocumentCommand{\potcrtajCvor}{O{0.4} m O{0}}{%#
% Potcrtavanje cvorova (zatvaranje grana tabloa)
% \newcommand{\potcrtajCvor}[1]{%
% #1 optional  -- prosirivanje crte
% #2 ime cvora
% #3 spustanje crte... (negativan broj)
\draw[ultra thick] (#2.south west) +(-#1,#3) -- ($(#2.south east) +(#1,#3)$);
% let p1 = (#1.south - #1.east)
}


% labela lijevo od D
\newcommand{\desnaLabela}[2]{%
\node at (#2 -| mr) {$(#1)$};
}

\newcommand{\lijevaLabela}[2]{%
\node at (#2 -| lr) {$(#1)$};
}

\newcommand{\rDesnaLabela}[2]{%
\node  at (#2 -| mr) {#1};
}



\newcommand{\minfup}[2]{%
\displaystyle {{#1} \atop {\phantom{#2}}}%
}

\newcommand{\minfdown}[2]{%
\displaystyle {\phantom{#1} \atop {#2}}%
}




% >>>>>>>>>>>>>>>>>>>>>>>>>>>>>>>>>>>>>>>>>>>>
%     \usepackage{showidx} % ili % \showlabels{index}

% \usepackage[right]{showlabels,rotating}  % \usepackage[notref]{showkeys} % obsolete by 'showlabels'
% \renewcommand{\showlabelsetlabel}[1]{\begin{turn}{80} \showlabelfont \smaller\smaller\smaller\smaller%
% \ttfamily #1\end{turn}}%
% 
% 
% \showlabels{index}


\setlength{\parindent}{0pt}
\setlength{\parskip}{2ex}
% -------------  OLD
% \linespread{1.2}  % ?
% \linespread{1.3}  % <-- PhD formalni zahtjevi (ublazeni)
% \linespread{1.5}  % <-- PhD formalni zahtjevi
    
\newcommand{\RLC}{%
\ensuremath{\mathrel{|\!\!\!\sim}}%
}
