%Napomena: linkovi i animacije kod prijelaza sa slajda na slajd funkcioniraju samo u PDF-u.
\documentclass[%
%a4paper,
fyma,pdf,colorBG,slideColor]{prosper}


% \hypersetup{pdfpagemode=FullScreen}%


%
\usepackage[croatian]{babel}%
\usepackage{amsmath}%
\usepackage{graphicx}%
%



\title{Moja nova \LaTeX prezentacija}
	\subtitle{ podnaslov }
	\author{Ivan Ivanovi\'c, Petar Petrovi\'c}
	\email{ime1@tug.hr, ime2@tug.hr}
	\institution{\href{http://www.foi.hr/}{Moja institucija}}

%
%  kod prezentacija su novost prijelazi (transitions). to
%  treba protumaciti 
%

\begin{document}
\maketitle

\begin{slide}%[Box]
{Apsolutna vrijednost}
Definiramo funkciju $|\ \ |:\mathbb{R}\rightarrow \mathbb{R}$ sa
$$
|x|=\begin{cases}
\hphantom{-} x,  & \text{ako je } x\geqslant 0\\
            -x,  & \text{ako je } x<0
\end{cases}
$$
koju zovemo {\red apsolutna vrijednost}. Kliknite: 
\hyperlink{RAVNINA}{\blue ravnina}. \hypertarget{APS}{}
\end{slide}



\overlays{5}{
\begin{slide}[Box]{Svojstva apsolutne vrijednosti}
\untilSlide{3}{Apsolutna vrijednost ima sljede\'ca svojstva:}
\begin{itemstep}
\item $|x|\geqslant 0$
\item $|x|=0\Leftrightarrow x=0$
\item $|xy|=|x||y|$
\item $\big|\frac{x}{y}\big|=\frac{|x|}{|y|}$
\item $|x+y|\leqslant |x|+|y|$
\end{itemstep}
\onlySlide{5}{za sve $x,y\in\mathbb{R}$}
\end{slide}
}




\begin{slide}[Box]{Ravnina}
$\Pi\ldots T_0(x_0,y_0,z_0),\ \vec{a}=(a_x,a_y,a_z),\ \vec{b}=(b_x,b_y,b_z)$\\[3pt]
\hypertarget{RAVNINA}  Kliknite: \hyperlink{APS}{\blue apsolutna vrijednost}
\end{slide}

\end{document}
