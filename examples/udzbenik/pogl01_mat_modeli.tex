\chapterwithquote{Matemati\v{c}ki modeli i struktura matematike}
{``Inteligencija ne mo\v{z}e biti prisutna bez razumijevanja.
Ra\v{c}unalo nema svijest o tome \v{s}to radi''\\[1ex]
Roger Penrose\footnote{(r.~1931.) poznati engleski matemati\v{c}ar sa zna\v{c}ajnim radovima u kozmologiji, algebri i geometriji}}
%\FIXME{Kamo sa fusnotama iz naslova?}

\footnotetext{Roger Penrose (r.~1931.) - poznati engleski matemati\v{c}ar
sa zna\v{c}ajnim radovima u kozmologiji, algebri i geometriji}

\section{Matemati\v{c}ki modeli}

\subsection{Modeli}



\FIXME{TH kaze da ovaj par ide van} Osnova za razumijevanje
svijeta je promatranje. Promatranjem prikupljamo informacije. Na
temelju pojedina\v{c}nih informacija radimo generalizacije,
naj\-prije jednostavne, a onda dolazimo do razumijevanja na
temelju principa. Princip je poop\'{c}enje ili apstraktna tvrdnja.

%Model je

%\FRAMEfhF}{7.5602in}{1.9406in}{0pt}{}{}{Figure}{\special{language
%"Scientific Word";type "GRAPHIC";display "USEDEF";valid_file
%"T";width 7.5602in;height 1.9406in;depth 0pt;original-width
%9.1073in;original-height 2.5244in;cropleft "0";croptop
%"1";cropright "1";cropbottom "0";tempfilename
%'HQX9F90N.wmf';tempfile-properties "PR";}}

Jedan od na\v{c}ina da se odgovori na pitanja koja se postavljaju u razli%
\v{c}itim znanstvenim podru\v{c}jima ili da se rije\v{s}i neki problem, je
konstrukcija odgovaraju\'{c}eg modela. Zbog toga se znanstvena metoda u prou%
\v{c}avanju razli\v{c}itih fenomena u su\v{s}tini svodi na kreiranje,
verifikaciju i stalno modificiranje razli\v{c}itih modela s ciljem da se
pojednostavni i objasni kompleksnost onoga \v{s}to se promatra i na temelju
toga kao kona\v{c}an cilj predvide i kontroliraju razli\v{c}iti
procesi.
\bigskip

Pojam model koristi se u razli\v{c}itim kontekstima tako da
ponekad i gubi svoj izvorni smisao (npr. kada se govori o fotomodelu, ili se govori o
modelu automobila i sl.). Su\v{s}tina pojma
model je ta da on predstavlja zamjenu za neki realni objekt ili
pojavu. Mo\v{z}emo re\'ci, model je    analogija s nekim objektom
ili drugim interesantnim modelom, a koristi se za obja\v{s}njenje
nekog procesa ili predvi\dj{}anje doga\dj{}aja.

\subsection{Svrha modela}

Modeli imaju razli\v{c}itu namjenu; s lutkom koja je model ljudskog bi\'{c}a
djeca se igraju (naravno, psiholog \'{c}e re\'{c}i da ona u\v{c}e), s ve\'{c}%
om lutkom mo\v{z}e se uvje\v{z}bavati davanje umjetnog disanja, dje\v{c}ak
\'{c}e se s malim brodi\'{c}em igrati, a znanstvenik u institutu za
brodogradnju \'{c}e na temelju pona\v{s}anja modela u bazenu poku\v{s}ati
predvidjeti pona\v{s}anje broda odgovaraju\'{c}ih karakteristika u realnim
uvjetima. Spomenuti modeli su materijalni modeli i naj\v{c}e\v{s}\'{c}e
predstavljaju umanjene replike stvarnih objekata. Me\dj utim, modeli ne
moraju imati fizi\v{c}ku sli\v{c}nost s objektom koji je predmet pa\v{z}nje.
U kemiji smo upoz\-na\-li modele atoma i molekula koji su bili svedeni na
raznobojne kuglice povezane \v{s}tapi\'{c}ima. Za obja\v{s}njenje
jednostavnijih fizikalnih zakona tako\dj er smo koristili materijalne modele
(npr. gibanje po kosini, njihala i sl.) da bi zatim konstruirali apstraktni
matemati\v{c}ki model za obja\v{s}njenje promatrane pojave. Spomenuti modeli
imaju svoju ulogu u prezentaciji nekog efekta. Me\dj utim, ukoliko se \v{z}%
eli neki fenomen objasniti do te mjere da se mo\v{z}e to\v{c}no
predvidjeti budu\'{c}e stanje sustava s kojim je on povezan,
moramo se poslu\v{z}iti slo\v{z}enijim modelima. Na primjer,
kretanje planeta u Sun\v{c}evom sustavu mo\v{z}emo prikazati
skicom ili \v{c}ak konstruirati fizi\v{c}ki model pomo\'{c}u kojeg
se mogu obja\v{s}njavati odnosi izme\dj{}u Sunca i planeta, ali da
bi se predvidjela pozicija pojedinog planeta u odre\dj enom dijelu
godine, potrebno je koristiti odgovaraju\'{c}i matemati\v{c}ki
model.

\subsection{Matemati\v{c}ki modeli}

Matemati\v{c}ki model sadr\v{z}i sljede\'{c}e bitne komponente;
\textbf{pojavu ili proces} iz realnog svijeta koji se \v{z}eli
modelirati, \textbf{apstraktnu matemati\v{c}ku strukturu} i
\textbf{korespondenciju} izme\dj{}u elemenata prve i druge
komponente. Realnost opisujemo objektima,
parametrima, vezama i doga\dj ajima. Tim pojmovima pridru\v{z}uju se matemati%
\v{c}ki pojmovi iz apstraktne matemati\v{c}ke strukture, varijable, relacije
me\dj u matemati\v{c}kim pojmovima i operacije s njima. Matemati\v{c}ki
modeli temelje se na razli\v{c}itim pretpostavkama o realnom sustavu ili
fenomenu koji se prou\v{c}ava, a one se reprezentiraju jednad\v{z}bama,
nejednad\v{z}bama, funkcijama i drugim matemati\v{c}kim pojmovima u kojima
se pojavljuju razli\v{c}ite varijable i parametri. Najjednostavniji matemati%
\v{c}ki modeli su funkcije koje reprezentiraju povezanost dviju ili vi\v{s}e
varijabli.

Kod modeliranja se postavlja pitanje odnosa izme\dj u
slo\v{z}enosti modela i njegove upotrebljivosti. Slo\v{z}enost
modela karakterizirana je prvenstveno brojem varijabli koje se
nastoje povezati i matemati\v{c}kim svojstvima veza izme\dj u
njih. Treba te\v{z}iti \v{s}to jednostavnijem modelu, ali svako
pojednostavljivanje modela povezano je s pove\'{c}anjem razine
apstrakcije i time se smanjuje mogu\'{c}nost primjene rezultata
modela u obja\v{s}njavanju fenomena koji se modelira. S druge
strane, nastojanje da se koristimo slo\v{z}enim modelom povezano
je s problemima prikupljanja dovoljne koli\v{c}ine podataka,
problemima rje\v{s}ivosti modela i mogu\'{c}no\v{s}\'{c}u da se
kvalitetno interpretiraju i prezentiraju rezultati takve analize.

Matemati\v{c}ki modeli op\'{c}enito sadr\v{z}e tri razli\v{c}ite vrste
kvantitativnih veli\v{c}ina; izlazne varijable (output), ulazne varijable
(input) i parametre (konstante). Vrijednosti izlaznih varijabli \v{c}ine rje%
\v{s}enje modela. Izbor ulaznih varijabli i parametara u domeni je tvorca
modela i taj izbor u najve\'{c}oj mjeri odre\dj uje kvalitetu i slo\v{z}%
enost modela.

Op\'{c}a upotrebljivost matemati\v{c}kog modela mo\v{z}e se objasniti preko
svojstava koja se i ina\v{c}e povezuju s matemati\v{c}kim karakterizacijama.
Ta svojstva su:

\textit{formalizacija} - matemati\v{c}ke \ relacije omogu\'{c}uju jasno
razumijevanje odnosa izme\dj u dijelova promatranog sustava i njegovo
funkcioniranje,

\textit{preciznost} - poznato je da matematika daje precizan rezultat,
odnosno to\v{c}nije re\v{c}eno, zna se u kojoj mjeri je rezultat primjene
odre\dj enog modela precizan. Za situacije kada se modeliraju pojave s
nesigurno\v{s}\'{c}u, postoje statisti\v{c}ke metode koje u toj nesigurnosti
identificiraju skrivene veze i omogu\'{c}uju njezino mjerenje,

\textit{fleksibilnost} - matemati\v{c}ki modeli se temelje na pretpostavkama
i sadr\v{z}e parametre koji omogu\'{c}uju prilago\dj avanje modela
promjenama u realnom sustavu,

\textit{mogu\'{c}nost provjere i predvidljivost} - matemati\v{c}ki modeli su
jasni i odre\dj eni u tolikoj mjeri da se mogu provjeriti i omogu\'{c}avaju
da se predvide rezultati njihove primjene,

\textit{ekonomi\v{c}nost} - matematika je koncizna, ona nikada ne koristi vi%
\v{s}e alata nego je to potrebno.

Bitna prednost matemati\v{c}kih modela u odnosu na materijalne je ta da se
na simboli\v{c}kom modelu lak\v{s}e provode promjene nego na
materijalnom. Mijenjanjem parametara u modelu model se transformira i prilago%
\dj ava opa\v{z}anjima. Na\v{c}ini li se npr. matemati\v{c}ki model brodskog
trupa on \'{c}e sadr\v{z}avati parametre koji karakteriziraju njegove
dimenzije, kroz odnose dimenzija pojedinih djelova modeliraju se specifi\v{c}%
ne karakteristike oblika trupa i takav jedan model u biti predstavlja mno%
\v{s}tvo modela. S takvim modelom daleko je lak\v{s}e ispitati pona\v{s}anje
budu\'{c}eg broda i tra\v{z}iti najbolji oblik trupa nego graditi mno\v{s}%
tvo materijalnih modela i kupati ih u bazenu za ispitivanje. Osim toga,
postoje brojni vrlo op\'{c}eniti matemati\v{c}ki modeli koji se mogu lako
adaptirati u razli\v{c}itim realnim situacijama. Na primjer, linearna
funkcija predstavlja op\'{c}i model za mnoge ekonomske pojave, a normalna
krivulja se koristi u obja\v{s}njavanju mnogih problema u dru\v{s}tvenim
znanostima. Osim toga, u mnogim matemati\v{c}kim modelima mogu\'{c}e je
izvesti transformacije koje se mogu interpretirati kao promjene u sustavu
ili procesu koji se modelira.

Vi\v{s}e o modeliranju mo\v{z}e se na\'ci u \cite{stanat:mcalister}.

\subsection{Svrha matemati\v{c}kih modela}

Mogu se nabrojiti razli\v{c}iti motivi za razvoj modela ali mi \'{c}emo se
ograni\v{c}iti na tri temeljna koji se odnose na matemati\v{c}ke modele.
\begin{enumerate}[(i)]

\item Prezentiranje informacija u \v{s}to razumljivijem obliku

Dobri primjeri za ovakve modele su plan grada i zemljopisna karta. Uz malo
znanja o simbolima koji se koriste u njima, iz tih prikaza mogu se dobiti
bitne informacije za orijentaciju u prostoru. Gledano matemati\v{c}ki,
zemljovidi su grafovi. Iako se uz malo dodatnog truda iz informacija koje
daje zemljovid mogu izvesti brojni zaklju\v{c}ci, te\v{s}ko se mogu dobiti
eksplicitni odgovori na pitanja poput: "Kojim putem i\'{c}i od to\v{c}ke A
do to\v{c}ke B u vrijeme prometne \v{s}pice?" ili "Kako u najkra\'{c}em
vremenu obi\'{c}i odre\dj ene gradove?". U tra\v{z}enju odgovora na ta i sli%
\v{c}na pitanja poma\v{z}e posebna matemati\v{c}ka disciplina, teorija
grafova.

\item Jednostavnije ra\v{c}unanje

Mnogi prakti\v{c}ni problemi mogu se rije\v{s}iti uz primjenu jednostavnijih
matemati\v{c}kih postupaka op\'{c}e namjene, ali uz cijenu dugotrajnog ra%
\v{c}unanja i manje to\v{c}nosti. Me\dj utim, razvoj posebnih matemati\v{c}%
kih modela omogu\'{c}uje br\v{z}e dola\v{z}enje do rezultata i kvalitetniju
analizu problema. Tako npr.~modeli linearnog programiranja omogu\'{c}uju da
se izradi plan proizvodnje s ciljem optimalizacije profita (ili
minimalizacije tro\v{s}kova proizvodnje).

\item Predvi\dj anje

Tre\'{c}a svrha matemati\v{c}kih modela je da se pomo\'{c}u njih predvide
budu\'{c}a stanja sustava koji se modelira ili na\v{c}in odvijanja nekog
procesa. Takav je npr. matemati\v{c}ki model kojim se nastoji predvidjeti
pona\v{s}anje broda odgovaraju\'{c}ih karakteristika. Poznat je primjer da
je pomo\'{c}u matemati\v{c}kog modela otkriven planet Neptun na temelju uo%
\v{c}enih odstupanja u o\v{c}ekivanoj orbiti planeta Urana. Vremenske
prognoze temelje se na obradi velikog broja podataka pomo\'{c}u slo\v{z}enih
matemati\v{c}kih modela. Postoji posebna disciplina koja se bavi razvojem
razli\v{c}itih prognosti\v{c}kih modela. Ti modeli daju odgovore na pitanja
o o\v{c}ekivanom smjeru poslovnih doga\dj aja, a razvijeni su i modeli za
prognoziranje kretanja vrijednosti dionica na burzama, modeli za
prognoziranje u\v{c}estalosti nesretnih doga\dj aja (za potrebe osiguranja
od \v{s}teta), i drugi. Kod predvi\dj anja se postavlja pitanje to\v{c}nosti
s kojom se mo\v{z}e predvidjeti neki doga\dj aj ili pojava. Matemati\v{c}ki
modeli koji se temelje na fizikalnim zakonima uglavnom omogu\'{c}uju to\v{c}%
no predvi\dj anje (npr. to\v{c}no se mo\v{z}e odrediti vrijeme nastupanja
pomr\v{c}ine nekog nebeskog tijela, putanja lansiranog svemirskog broda i
sl.). S druge strane pak, za sada se bez obzira na slo\v{z}enost matemati%
\v{c}kog modela, ne mo\v{z}e sa sigurno\v{s}\'{c}u predvidjeti budu\'{c}e
stanje ekonomije na temelju mjera ekonomske politike koje se mogu poduzeti.
Sli\v{c}an slu\v{c}aj je i s vremenskim prognozama.
\end{enumerate}

\subsection{Matemati\v{c}ko modeliranje}

Matemati\v{c}ko modeliranje je proces matemati\v{c}ke reprezentacije nekog
fenomena s ciljem njegovog boljeg razumijevanja. Pri tom va\v{z}nu ulogu igra
postupak apstrakcije. Apstrakcija se u su\v{s}tini svodi na to da se
prepoznaju elementi koji nisu toliko bitni za funkcioniranje sustava koji se
modelira i da se oni zanemare kod kreiranja modela.

Izgradnja matemati\v{c}kog modela mo\v{z}e se objasniti u nekoliko koraka:
\begin{enumerate}

\item Pojednostavljivanje (apstrakcija)- u sustavu ili procesu
koji se modelira nastoje se prepoznati bitni elementi, a ostali se
zanemaruju.

\item Prikaz (reprezentacija) - elementima sustava ili procesa
pridru\v{z}uju se matemati\v{c}ki simboli, a odnosima me\dj u
elementima pridru\v{z}uju se (ne)jednad\v{z}be.

\item Transformacije - rje\v{s}enje matemati\v{c}kog modela
potrebno je oblikovati i interpretirati u obliku koji predstavlja
odgovor na pitanje koje nas je i motiviralo na izgradnju modela.

\item Verifikacija - zaklju\v{c}ke izvedene u prethodnom koraku
potrebno je usporediti s rezultatima opa\v{z}anja sustava ili
procesa koji se modelira. Odstupanja su temelj za eventualnu
prilagodbu modela.
\end{enumerate}

\subsection{Podjela matemati\v{c}kih modela}

Model je deterministi\v{c}ki ukoliko se razvija direktno na temelju
fizikalnih zakona. Takvi modeli se koriste npr. u slu\v{c}aju kada se \v{z}%
eli odrediti putanja po kojoj raketa treba letjeti na mjesec. Za prognozu
vremena potrebno je razvijati modele koji se temelje na empirijskim
podacima, a rezultati koje daju takvi modeli sadr\v{z}e odre\dj enu razinu
nesigurnosti. Takvi modeli nazivaju se stohasti\v{c}ki.

Matemati\v{c}ki modeli dijele se i po drugim kriterijima, a osnovne podjele
temelje se na matemati\v{c}koj strukturi koja se koristi u modeliranju. Tako
npr. govorimo o linearnom modelu ako su sve jednad\v{z}be i funkcije koje
se javljaju u modelu linearne. Podjela se mo\v{z}e temeljiti i na specifi%
\v{c}nostima varijabli i parametara u modelu. To je posebno nagla\v{s}eno u
matemati\v{c}kim modelima procesa koji se odvijaju u vremenu; razlikuju se
modeli u kontinuiranom vremenu i mo\-de\-li s diskretnim vremenom. Oba ova
modela imaju svoje prednosti i nedostatke. Rje\v{s}enja modela s
kontinuiranim vremenom daju nam informacije o promatranom fizikalnom
fenomenu u nekom neprekidnom intervalu vremena (kontinuumu) za razliku od
modela s diskretnim vremenom koji obja\v{s}njavaju pona\v{s}anje sustava u
odre\dj enim vremenima. Prednost prvih modela nad drugima je sa stajali\v{s}%
ta primjenjivosti rezultata o\v{c}ita; oni omogu\'{c}avaju da se pona\v{s}%
anje promatranog sustava kontrolira kroz \v{c}itavo vrijeme i jasnije
pokazuju efekte promjene vrijednosti ulaznih varijabli i parametara. S druge
strane pak modeli s diskretnim vremenom imaju tu prednost da je za njihov
razvoj potrebno poznavati jednostavnije matemati\v{c}ke discipline i da su
pogodniji za ra\v{c}unarsku primjenu. Ve\'{c}ina modela u ra\v{c}unarskim i
informacijskim znanostima je upravo ovog drugog tipa.

\bigskip


\FIXME{}


\section{Kako se gradi matemati\v{c}ka teorija}

%\subsubsection{Kako se gradi matemati\v{c}ka teorija}

\subsubsection{Matemati\v{c}ki pojmovi}

Osnovni elementi svake matemati\v{c}ke teorije su \textbf{matemati\v{c}ki
pojmovi}. Oni se dijele na \textbf{osnovne} i \textbf{izvedene} (slo\v{z}%
ene) pojmove. Osnovni matemati\v{c}ki pojmovi naj\v{c}e\v{s}\'{c}e su
apstrakcija objekata ili pojmova iz stvarnog svijeta. Dobre primjere za obja%
\v{s}njenje matemati\v{c}kih pojmova imamo u geometriji. Pojam \textbf{pravac%
} nastao je apstrakcijom predmeta poput ravne niti, brida ravne plohe ili sun%
\v{c}eve zrake. Pojam \textbf{ravnina} je vrlo vjerojatno nastao
apstrakcijom iz ravnih povr\v{s}ina poput pustinje, povr\v{s}ine jezera ili
mora. Pod osnovne matemati\v{c}ke pojmove svrstavaju se i osnovni odnosi me%
\dj u pojmovima poput: \textbf{pripadati}, \textbf{sje\'{c}i}, \textbf{%
spajati,} \textbf{le\v{z}ati (u)}, te \textbf{izme\dj u}, \textbf{sukladno}
i \textbf{usporedno}. Me\dj u ovim pojmovima mogu se uvesti jo\v{s}
detaljnije podjele. Osnovni matemati\v{c}ki pojmovi nisu dovoljni da bi se
izgradila matemati\v{c}ka teorija. Pomo\'{c}u njih definiraju se slo\v{z}eni
pojmovi. Npr. trokut se definira kao dio ravnine ome\dj en s tri du\v{z}ine.

\subsubsection{Dokazivanje teorema}

Na temelju opa\v{z}anja i iskustva uo\v{c}avaju se zakonitosti me\dj u
odnosima koji vladaju izme\dj u predmeta stvarnog svijeta i zatim se te
zakonitosti nastoje oblikovati kao tvrdnje (teoremi, pou\v{c}ci) o odnosima
me\dj u odgovaraju\'{c}im matemati\v{c}kim pojmovima. Tvrdnje koje se izri%
\v{c}u moraju imati univerzalnu vrijednost koja se \textbf{dokazuje}. Za
postupak dokazivanja matemati\v{c}kih tvrdnji va\v{z}an je proces \textbf{%
zaklju\v{c}ivanja}. Pod zaklju\v{c}ivanjem se podrazumjeva takav oblik mi%
\v{s}ljenja kojim se vi\v{s}e tvrdnji dovodi u vezu i izvodi nova tvrdnja.
Dokazati neku tvrdnju zna\v{c}i pokazati da je ta tvrdnja logi\v{c}ka
posljedica nekih tvrdnji za koje se zna da su istinite. Uspije li se to
pokazati na takav na\v{c}in da se odabrani skup polaznih tvrdnji
transformira u logi\v{c}ki ekvivalentne tvrdnje sve dok se ne dobije tvrdnja
koja se dokazuje, govori se o \textbf{direktnom dokazu}. U matematici se \v{c}%
esto koristi i \textbf{indirektan dokaz} \v{c}ija logi\v{c}ka utemeljenost
\'{c}e biti obja\v{s}njena kasnije u poglavlju o matemati\v{c}koj logici.
Znanost koja izu\v{c}ava procese zaklju\v{c}ivanja zove se \textbf{logika},
a osnovni oblici zaklju\v{c}ivanja su \textbf{analogija}, \textbf{indukcija}
i \textbf{dedukcija}. Svi oblici zaklju\v{c}ivanja nisu jednako va\v{z}ni za
matematiku.

Analogija je takav na\v{c}in zaklju\v{c}ivanja pri kojem se na temelju uo%
\v{c}enih zakonitosti u odre\dj enoj situaciji izvodi
zaklju\v{c}ak koji bi trebao vrijediti u nekoj drugoj situaciji.
Ovaj na\v{c}in zaklju\v{c}ivanja nije pouzdan i mo\v{z}e dovesti
do potpuno krivih zaklju\v{c}aka. U matematici analogija mo\v{z}e
pomo\'{c}i da se na temelju rezultata iz jednog podru\v{c}ja
matematike poku\v{s}a razviti teorija koja bi vrijedila u nekom
drugom podru\v{c}ju, ali \v{c}injenica da je nova tvrdnja
``sli\v{c}na'' nekoj tvrdnji koja vrijedi nema snagu dokaza.

Indukcija je takav oblik zaklju\v{c}ivanja pri kojem se zaklju\v{c}ak o
ispravnosti neke op\'{c}e tvrdnje izvodi na temelju provjere o ispravnosti
te tvrdnje u posebnim slu\v{c}ajevima. U matematici vrijedi poseban oblik
indukcije tzv. potpuna indukcija. Vrijednost ovog na\v{c}ina zaklju\v{c}%
ivanja za matematiku je ograni\v{c}ena na dokazivanje nekih tvrdnji koje se
odnose na prirodne brojeve. Kasnije \'{c}emo pokazati kako se ova metoda
primjenjuje na konkretnim primjerima.

\subsubsection{Deduktivna metoda}

Od navedenih na\v{c}ina zaklju\v{c}ivanja najve\'{c}u va\v{z}nost za
matematiku ima dedukcija. Dedukcija se definira kao takav na\v{c}in zaklju%
\v{c}ivanja pri kojem se zaklju\v{c}ak o odnosima me\dj u matemati\v{c}kim
pojmovima u posebnim situacijama izvodi iz op\'{c}ih svojstava tih odnosa.
Mogu\'{c}nost da se u okviru neke teorije primijeni ovaj na\v{c}in zaklju%
\v{c}ivanja indikator je za visoku razinu razvoja te teorije. Primjenu
deduktivne metode u razvoju matemati\v{c}ke teorije karakteriziraju slijede%
\'{c}i koraci: (1) nabrajanje osnovnih pojmova, (2) definiranje slo\v{z}enih
pojmova, (3) izricanje aksioma, (4) postavljanje teorema, (5) dokazivanje
teorema. Prvi i drugi korak smo ve\'{c} komentirali. Koraci (3) i (4) su
povezani jer su i aksiomi i teoremi tvrdnje o odnosima me\dj u matemati\v{c}%
kim pojmovima. Razlika izme\dj u aksioma i teorema je ta \v{s}to su aksiomi
tvrdnje koje se ne dokazuju, ve\'{c} se smatraju to\v{c}nima po definiciji
ili ukoliko se radi o pojmovima pomo\'{c}u kojih se modeliraju pojave iz
realnog svijeta to\v{c}nima na temelju iskustva, a teoreme je potrebno
dokazivati. Pri izboru aksioma potrebno je po\v{s}tivati odre\dj ena na\v{c}%
ela; na\v{c}elo nezavisnosti, na\v{c}elo potpunosti i na\v{c}elo neproturje\v{c}nosti.
 Na\v{c}elo nezavisnosti tra\v{z}i da niti jedan aksiom ne
smije biti izvediv iz preostalih. Na\v{c}elo potpunosti zahtijeva da bilo
koja tvrdnja unutar razmatrane teorije bude dokaziva ili oboriva na temelju
aksioma (direktno ili posredno uz pomo\'{c} prije dokazanih teorema). Na\v{c}%
elo neproturje\v{c}nosti zahtijeva da se na temelju izabranih aksioma ne mo%
\v{z}e dokazati ispravnost dviju tvrdnji koje su kontradiktorne. Iako ovi
zahtjevi izgledaju prirodno, neki rezultati iz matemati\v{c}ke logike
(G\"odel\footnote{Kurt G\"odel (1906--1978) - austrijski matemati\v{c}ar,
poznat po radovima vezanim uz aksiomatske matemati�ke sustave.}
je dokazao da je potpunost aksioma Peanove teorije
brojeva nedokaziva, tj.~postoje istine u teoriji brojeva koje se
ne mogu dokazati.) pokazuju da nije mogu\'{c} takav sustav aksioma koji bi
zadovoljio sva tri navedena na\v{c}ela.

Kao primjer primjene deduktivne metode u razvoju matemati\v{c}ke teorije i
dokaz da izbor aksioma nije jednostavan zadatak obi\v{c}no se spominje
euklidska geometrija. Euklid\footnote{(325 p.K. -- 265 p. K.)
 - najpoznatiji anti�ki matemati�ar, poznat po znamenitom djelu \emph{Elementi}.}
  je oko 300.g.p.K. uo\v{c}io da se tada%
\v{s}nje poznavanje geometrije mo\v{z}e sistematizirati i oblikovati kao
teorija koja po\v{c}iva na pet aksioma. Sve do polovice devetnaestog stolje%
\'{c}a brojni matemati\v{c}ari poku\v{s}avali su dokazati da je jedan od tih
aksioma (to\v{c}nije, peti - tzv. aksiom o paralelama) izvediv iz
preostalih. Iako u tome nisu uspjeli, njihov trud ipak nije bio uzaludan.
Indirektno, ti su napori doveli do razvoja neeuklidske geometrije koja,
pojednostavljeno re\v{c}eno, po\v{c}iva na sustavu aksioma u kojem je jedan
od njih negacija petog Euklidovog aksioma.
















%
%
%Pojmovi u matematici se dijele na osnovne i izvedene. Osnovni pojmovi se ne
%definiraju, a izvedeni se pojmovi definiraju pomo\'{c}u osnovnih. Definicija
%je sud pomo\'{c}u kojeg se odre\dj{}uje sadr\v{z}aj nekog pojma. Isti se
%pojam mo\v{z}e definirati na vi\v{s}e ekvivalentnih na\v{c}ina.
%
%Osnovni matemati\v{c}ki pojmovi su generalizacija objekata iz stvarnog
%svijeta (npr. pravac je generalizacija zrake svjetlosti).
%
%Prilikom formiranja tvrdnji koristimo se zaklju\v{c}ivanjem. Zaklju\v{c}%
%ivanje je na\v{c}in mi\v{s}ljenja kojim se vi\v{s}e sudova (premisa) dovodi
%u vezu i izvodi novi sud (zaklju\v{c}ak, rezultat). Razlikujemo induktivno i
%deduktivno zaklju\v{c}ivanje, te zaklju\v{c}ivanje po analogiji. Matemati%
%\v{c}ki su oblici induktivno i deduktivno zaklju\v{c}ivanje.
%
%Indukcija mo\v{z}e biti potpuna (matemati\v{c}ka) i nepotpuna.
%
%Deduktivna metoda karakterizira vi\v{s}i nivo razvoja neke znanosti.
%
%Euklid \ (ro\dj\ oko 365. p. K.) je primjenio deduktivne metode u
%geometriji. Geometrija \ je na taj na\v{c}in aksiomatizirana
%(vidi: S. Mintakovi\'{c}, \textit{Neeuklidska geometrija
%Loba\v{c}evskog}, \v{S}k. knj. Zg, 1972, \v{Z}. Dadi\'{c},
%\textit{Povijest ideja i metoda u matematici i fizici}, \v{S}k.
%knj. 1992.). Aritmetika je aksiomatizirana znatno kasnije, u
%drugoj polovici 19. st (vidi: Z. \v{S}iki\'{c}, \textit{Kako je
%stvarana novovjekovna matematika}, \v{S}k. knj., Zg, 1989).
%
%Bitni elementi matemati\v{c}ke teorije kod deduktivnog pristupa su:
%
%\begin{enumerate}
%\item Nabrajanje osnovnih pojmova,
%\item Definiranje slo\v{z}enih pojmova,
%\item Postavljanje aksioma,
%\item Izno\v{s}enje teorema,
%\item Dokazi teorema.
%\end{enumerate}
%
%
%Aksiomi i teoremi (pou\v{c}ci, stavci) izri\v{c}u ekvivalentne tvrdnje i
%iznose zaklju\v{c}ke o matemati\v{c}kim pojmovima i njihovim me\dj usobnim
%odnosima i vezama.
%
%Aksiomi su tvrdnje koje smatramo istinitima bez posebnog dokaza.
%
%Teoremi su tvrdnje koje logi\v{c}ki izviru iz aksioma.
%
%Svaki teorem treba izvesti (deducirati, dokazati) iz jednog ili vi\v{s}e
%aksioma u kona\v{c}no mnogo koraka.
%
%Dokaz je zaklju\v{c}ivanje kojim se pokazuje da je neki teorem logi\v{c}ka
%poslje\-dica nekih aksioma ili ve\'{c} dokazanih teorema. Dokaz mo\v{z}e biti
%direktan i indirektan.
