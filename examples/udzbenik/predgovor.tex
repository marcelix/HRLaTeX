\begin{preface}{Predgovor}


Ovaj ud\v{z}benik prvenstveno je namijenjen studentima kojima
matematika nije primarno podru\v{c}je izu\v{c}avanja, ve\'c im
slu\v{z}i kao alat za opis i razumijevanje problema iz drugih
znanstvenih podru\v{c}ja. Tako\dj{}er se o\v{c}ekuje da
savladavanje izlo\v{z}enog gradiva doprinese njihovoj sposobnosti
logi\v{c}kog i kreativnog mi\v{s}ljenja te sustavnog pristupa u
rje\v{s}avanju problema.

U skladu s tim ciljevima izabrane su teme koje se prvenstveno
odnose na studij informatike i ostalih dru\v{s}tvenih znanosti.

Da bi se studentima omogu\'cilo lak\v{s}e \v{c}itanje i usvajanje
znanja, svaka tema popra\'cena je rije\v{s}enim primjerima i
dodatnim obja\v{s}njenjima.

Svako poglavlje zapo\v{c}inje  kra\'cim tekstom u kojem bi
studentima trebali prona\'ci motivaciju i opravdanje za
izu\v{c}avanje odre\dj{}enih matemati\v{c}kih tema.

Na kraju svakog poglavlja nalazi se \textbf{dodatak}, koji se
naj\v{c}e\v{s}\'ce sastoji od tri dijela. U prvom dijelu nalaze se
\textbf{naprednije teme} ili dokazi tvrdnji povezani s tim
poglavljem, koji su  stavljeni u dodatak jer nisu nu\v{z}ni za
razumijevanje iznesenog gradiva, pa njihovo izdvajanje omogu\'cava
jednostavnije izno\v{s}enje i razumijevanje gradiva. S druge
strane, zbog matemati\v{c}ke korektnosti treba ih staviti u
ud\v{z}benik.

Svakodnevno iskustvo pokazalo je da je u gotovo svakom poslu koji
zahtjeva visoku stru\v{c}nu spremu, va\v{z}no znati pravilno
govoriti i pisati, uva\v{z}avaju\'{c}i logi\v{c}ku argumentaciju.
To zahtjeva poznavanje metoda istra\v{z}ivanja podataka, prikaza
podataka u matemati\v{c}ki ispravnom obliku, te kreativnost u
njihovoj upotrebi i interpretaciji. Zbog toga   u   ud\v{z}beniku
posebnu va\v{z}nost posve\'{c}ujemo rje\v{s}avanju problema i
matemati\v{c}koj pismenosti. Stoga na kraju svakog poglavlja
dajemo nekoliko prijedloga za \textbf{pismene radove iz
matematike} u obliku eseja ili problemskih tema. U prvom dijelu
ud\v{z}benika uglavnom se predla\v{z}u jednostavnije teme,
narativnog karaktera, koje dopunjuju ili pro\v{s}iruju
izlo\v{z}eno gradivo. Time se \v{z}ele posti\'ci sljede\'ci
ciljevi: da student nau\v{c}i pronalaziti primjerenu literaturu,
da se zna njome slu\v{z}iti, da razumije (matemati\v{c}ki) tekst,
da zna razlu\v{c}iti bitno od nebitnog, te da rad formulira
gramati\v{c}ki, stilski i matemati\v{c}ki ispravno na dvije A4
stranice teksta (maksimalno 6000 karaktera broje\'{c}i i razmake).
\FIXME{karaktera? mozda 'znakova'} Nasuprot tome, u drugom dijelu
ud\v{z}benika idemo korak dalje pa pretpostavljamo da je student
savladao nabrojene ciljeve i da je spreman rje\v{s}avati
problemske zada\'{c}e. Student radi na rje\v{s}avanju
specifi\v{c}nog problema te njegovoj obradi i prikazu u obliku
kra\'{c}eg eseja, kao i u prvom dijelu knjige. Ovdje je naglasak
na kreativnom radu i istra\v{z}ivanju primjerenom odre\dj enom stupnju ste%
\v{c}enog znanja.

U dodatku se nalaze i \textbf{zadaci za ponavljanje} gradiva
iznesenog u poglavlju. Rje\v{s}enja zadataka nalaze se na kraju
knjige.

\medskip

 Zahvaljujemo   recenzentima \v{s}to su nam ukazali na
nedostatke u knjizi i sugerirali nam pobolj\v{s}anja.

Posebno   zahvaljujemo na\v{s}im suradnicima Damiru Horvatu i
mr.~Zlatku Erjavcu koji su pa\v{z}ljivo pro\v{c}itali tekst i
pomagali u njegovom ure\dj{}ivanju.

Tako\dj{}er   zahvaljujemo i Marcelu Mareti\'cu koji je
ure\dj{}ivao tekst i na\v{c}inio ve\'ci dio slika u knjizi.
\bigskip

U Vara\v{z}dinu, lipnja 2004. \hfill Autori


\end{preface}


\thispagestyle{plain}
