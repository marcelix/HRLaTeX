\documentclass{ppclanak}


\begin{document}


	
%%%%%%%%%%%%%%%%%%%%%%%%%%%%%%%%%%%%%%



\autortoc{Zvonimir \v{S}iki\'{c}}%
\clanak{Zvonimir \v{S}iki\'{c}}{Rje\v{s}enja pet zadataka iz
vjerojatnosti}

U Pou\v{c}ku br....\fixme{referenca} objavili smo 5 zadataka iz
vjerojatnosti koje ste, nadamo se, uspje\v{s}no rije\v{s}ili.
Ovdje nudimo na\v{s}a rje\v{s}enja u koja smo upleli dosta
zanimljive matematike (geometrijski red, Eulerovu konstantu
$\gamma$, beskona\v{c}no protegnutu povr\v{s}inu Nikole Oresma,
razvoj od $e^{-x}$ u red potencija itd.~). Ukoliko su va\v{s}a
rje\v{s}enja zanimljiva i bitno razli\v{c}ita, javite nam.

\begin{enumerate}
\item Ve\'{c}ina ljudi nekako osje\'{c}a da je odgovor 6. To je
jedan od rijetkih slu\v{c}ajeva u kojima nas vjerojatnosni
osje\'{c}aj ne vara.

Mi \'{c}emo dokazati ne\v{s}to op\'{c}enitiji rezultat.
Pretpostavimo da je vjerojatnost \v{z}eljenog ishoda nekog
eksperimenta $p$. Tada ne\v{z}eljeni ishod ima vjerojatnost
$q=1-p$. Vjerojatnost da se \v{z}eljeni ishod polu\v{c}i tek u
$n$-tom ponavljanju me\dj usobno nezavisnih eksperimenata je
$P_n=pq^{n-1}$ ($n-1$ puta mora se zbiti ne\v{z}eljeni ishod
\v{c}ija je vjerojatnost $q$, a zatim se jedanput mora zbiti
\v{z}eljeni ishod \v{c}ija je vjerojatnost $p$). O\v{c}ekivana
vrijednost potrebnog broja poku\v{s}aja zato je
\[ E= 1P_1 +  2P_2 + \ldots =\sum _{n=1}^{\infty} nP_n= p + 2pq + 3pq^2 + \ldots = \sum _{n=1}^{\infty}npq^{n-1} .\]

O\v{c}ekivanu vrijednost $E$ mo\v{z}emo izra\v{c}unati metodom
kojom se obi\v{c}no ra\v{c}una suma geometrijskog reda:
\begin{align*}
E =& p + 2pq + 3pq^2 + 4pq^3 + 5pq^4 + \ldots\\
qE =& \;\;\;\;\;\;\;\; pq + 2pq^2 + 3pq^3 + 4pq^4 + \ldots\\
E - qE=& p + \;\: pq + \;\: pq^2 + \;\: pq^3 + \;\: pq^4 + \ldots
\end{align*}
\fixme{Kakva je ovo formula?}

% \begin{align*}
% E =& p &+& 2pq &+& 3pq^2 &+& 4pq^3 &+& 5pq^4 &+& \ldots\\
% qE =& &pq &+& 2pq^2 &+& 3pq^3 &+& 4pq^4 &+& \ldots \\
% E - qE=& p &+& pq &+& pq^2 &+& pq^3 &+& pq^4 &+& \ldots
% \end{align*}
Primjenom formule za sumu geometrijskog reda sada lako nalazimo:
\[ E(1-q)=p(1+q +q^2 +q^3 + q^4 + \ldots) = \frac{p}{1-q} = \frac{p}{p}=1 . \]
Dakle, $E=1/(1-q)=1/p$. To zna\v{c}i da, u prosjeku, eksperiment
moramo ponoviti $1/p$ puta da bismo do\v{s}li do \v{z}eljenog
ishoda. Ako je $p=1/6$, kao u na\v{s}em zadatku, eksperiment
trebamo provjeriti prosje\v{c}no $6$ puta.

Isti problem mo\v{z}emo rije\v{s}iti i bez ra\v{c}unanja
beskona\v{c}nih suma. Uo\v{c}imo da prvi u nizu eksperimenata
mo\v{z}e dati ne\v{z}eljeni ishod (s vjerojatno\v{s}\'{c}u $q$)
ili \v{z}eljeni ishod (s vjerojatno\v{s}\'{c}u $p$). U prvom
slu\v{c}aju kre\'{c}emo ispo\v{c}etka, tj.~o\v{c}ekivani broj
eksperimenata do \v{z}eljenog ishoda sada je $1+E$. U drugom
slu\v{c}aju do\v{s}li smo do \v{z}eljenog ishoda,
tj.~o\v{c}ekivani broj eksperimenata do \v{z}eljenog ishoda je 1.
Sada je
\[ E=(1+E)q +1p,\]
odakle odmah slijedi
\[ E(1-q)= q+ p =1, \;\;\;\; E=\frac{1}{1-q} = \frac{1}{p}.\]

\item U prvoj kutiji sigurno \'{c}ete na\'{c}i jedan broj.
Vjerojatnost da u sljede\'{c}oj kutiji na\dj ete novi broj iznosi
$4/5$. Prema rje\v{s}enju prethodnog zadatka, prosje\v{c}no
trebate kupiti $1/(4/5)=5/4$ kutije da bi se to desilo.
Vjerojatnost da u sljede\'{c}oj kutiji na\dj ete novi broj je
$3/5$ i u prosjeku trebate kupiti $1/(3/5)=5/3$ kutija da bi se to
desilo. Zadnja dva broja, u prosjeku, zahtijevaju kupnju
$1/(2/5)=5/2$ i $1/(1/5)=5$ kutija. Ukupan (prosje\v{c}ni) broj
kutija koje trebate kupiti je
\[ 1 + \frac{5}{4} + \frac{5}{3} + \frac{5}{2} + 5= 5\left(\frac{1}{5} + \frac{1}{4} + \frac{1}{3} +\frac{1}{2} + \frac{1}{1}\right)\approx 11.4 .\]
Kada bi DEX u svojoj nagradnoj igri koristio $n$ brojeva,
prosje\v{c}an broj kutija koji biste trebali kupiti bio bi
\[ n\left(1 + \frac{1}{2} +\frac{1}{3}+\ldots + \frac{1}{n}\right).\]
Tu vrijednost mo\v{z}emo lako izra\v{c}unati koriste\'{c}i se
aproksimacijom
\[ H_n= 1 +\frac{1}{2} +\frac{1}{3}+\ldots + \frac{1}{n} \approx \ln n + \gamma +\frac{1}{2n},\]
gdje je $\gamma$ slavna Eulerova konstanta,
$\gamma=0.577218\ldots$.

Na primjer, za $n=5$ nalazimo
\[ 5\left(1 + \frac{1}{2} +\frac{1}{3}+\frac{1}{4} + \frac{1}{5}\right)\approx 5\ln 5 +5\gamma +\frac{1}{2}\approx 11.4 \]
\v{s}to je, na jednu decimalu, ista vrijednost koju dobijemo i
direktnim ra\v{c}unanjem vrijednosti $5(1+1/2+\ldots +1/5)$.

Objasnit \'{c}emo kako mo\v{z}emo do\'{c}i do te aproksimacije, i
do same konstante $\gamma$. Krenimo od iznosa povr\v{s}ina na
sl......\fixme{referenca na sl.1, ubacit sl. 1...}

Ukupna povr\v{s}ina nezatamnjenih pravokutnika, \v{c}ije se
osnovice prote\v{z}u od 0 do $n$, ima iznos
\[ H_n= 1 +\frac{1}{2} +\frac{1}{3}+\ldots + \frac{1}{n}.\]
Isti iznos ima i ukupna povr\v{s}ina pravokutnika sa zatamnjenim
vrhovima, \v{c}ije se osnovice prote\v{z}u od 1 do $n+1$.
Povr\v{s}ina od osi $x$ do grafa $y=1/x$, nad intervalom $[1,n]$,
iznosi $\ln n$. Odavde, uz pokrate $\overline{\delta} _n=\delta _1
+\delta _ 2 + \ldots + \delta _{n-1}$ i $\overline{\gamma}
_n=\gamma _1 +\gamma _ 2 + \ldots + \gamma _{n-1}$, lako slijedi
(v.~sl.~ \fixme{ref na sliku 1})
\begin{equation}\label{Eq1}
H_n= 1 +\frac{1}{2} +\frac{1}{3}+\ldots + \frac{1}{n}= \ln{(n+1)}
- (\delta _1 +\delta _ 2 + \ldots + \delta _{n-1})= \ln{(n+1)} -
\overline{\delta} _{n},
\end{equation}
\begin{equation}\label{Eq2}
H_n= 1 +\frac{1}{2} +\frac{1}{3}+\ldots + \frac{1}{n}= \ln{(n+1)}
+ (\gamma _1 +\gamma _ 2 + \ldots + \gamma _{n})= \ln{(n+1)} +
\overline{\gamma} _{n+1}.
\end{equation}
Prika\v{z}emo li sve povr\v{s}ine $\gamma _1, \gamma _2,\ldots ;
\delta _1, \delta _2 ,\ldots$ u $1\times 1$ pravokutniku
(v.~sl.~\fixme{referenca na sliku 2} lako \'{c}emo se uvjeriti da
vrijede sljede\'{c}e tvrdnje: \fixme{ubacit sliku 2}
\begin{equation}\label{Eq3}
\sum _{n=1}^{\infty} \gamma _n = \lim _{n \rightarrow\infty}
{\overline{\gamma} _n} =\gamma > \frac{1}{2}, \;\;\;\; \sum
_{n=1}^{\infty} \delta _n = \lim _{n \rightarrow\infty}
{\overline{\delta} _n} =\delta < \frac{1}{2}, \;\;\;\; \gamma +
\delta = 1
\end{equation}

\begin{equation}\label{Eq4}
(\gamma -\overline\gamma _n ) + (\delta - \overline\delta _ n) =
\frac{1}{n}, \;\;\;\; \delta - \overline\delta _n
\approx\frac{1}{2n} \approx\gamma -\gamma _n
\end{equation}

Iz (\ref{Eq1}) i (\ref{Eq4}) sada lako slijedi
\begin{equation*}
H_n= 1+ \ln n -\overline\delta _n = 1+ \ln n + (\delta -\delta _n)
-\delta = \ln n + \gamma + (\delta - \delta _n)\approx \ln n +
\gamma + \frac{1}{2n} ,
\end{equation*}
\v{s}to je na\v{s}a aproksimacija. Osim toga, iz (\ref{Eq2}) i
(\ref{Eq4}) slijedi:

\begin{equation*}
\begin{split}
H_n &=\ln{(n+1)} +\overline\gamma _{n+1} = 1+ \ln{(n+1)}
+\overline\gamma _{n+1} -\gamma + \gamma \approx \ln{(n+1)} +
\gamma
-\\
& - \frac{1}{2(n+1)}\approx \ln{n} + \frac{1}{n+1} + \gamma -
\frac{1}{2(n+1)} = \ln n +\gamma -\frac{1}{2(n+1)}.
\end{split}
\end{equation*}

Naime, $\ln{(n+1)} -\ln n=1/(n+1) +\delta _n \approx 1/(n+1)$,
\v{s}to se lako vidi na sl.~\fixme{ref na sliku 3} \fixme{anpak
slika tri}

Sve u svemu, imamo sljede\'{c}u procjenu:
\[ \ln n +\gamma - \frac{1}{2(n+1)}<H_n<\ln n +\gamma +\frac{1}{2n} .\]

\item Vjerojatnost pisma je $1/2$. Vjerojatnost 2 pisma u niza je
$1/4$. Vjerojatnost 3 pisma u nizu je $1/8$. Vjerojatnost 4 pisma
u nizu je $1/16$. I tako dalje. Dakle, vjerojatnost bloka duljine
$n$ je $1/2^n$. Slijedi da je prosje\v{c}na duljina bloka
\[D=\frac{1}{2}\cdot 1 + \frac{1}{4}\cdot 2 + \frac{1}{8}\cdot 3 +\ldots +\frac{1}{2^n}\cdot n + \ldots\]

Ovaj beskona\v{c}ni zbroj mo\v{z}emo izra\v{c}unati na razne
na\v{c}ine. Pokazat \'{c}emo kako ga je (u sasvim drugom
kontekstu) izra\v{c}unao Nikola Oresme u 14.~st. On je konstruirao
sljede\'{c}u povr\v{s}inu koja predstavlja tu sumu: \fixme{nabac,
Marc, sliku4}

Zatim je tu povr\v{s}inu preslo\v{z}io na sljede\'{c}i na\v{c}in:
\fixme{obrni, deder, Marc -slika 5}

Dakle, $D=2$, tj.~prosje\v{c}na duljina bloka pisama je 2.

I ovdje mo\v{z}emo zaobi\'{c}i beskona\v{c}na zbrajanja. Zapitajmo
se na koliko promjena (iz glave u pismo ili iz pisma u glavu)
prosje\v{c}no nailazimo u nizu od $n$ bacanja nov\v{c}i\'{c}a.
Budu\'{c}i da je vjerojatnost promjene $1/2$, prosje\v{c}no
nailazimo na $n/2$ promjena. To zna\v{c}i da je prosje\v{c}na
duljina bloka 2. Uistinu jednostavnije rje\v{s}enje.

\item Ve\'{c}ina ljudi nekako osje\'{c}a da je vjerojatnost ista u
svim trima slu\v{c}ajima. (Prosje\v{c}an broj \v{s}estica na
\v{s}est ba\v{c}enih kocki je 1, na 12 je 2, na 18 je 3, a
$1:6=2:12=3:18$.) Na\v{z}alost, to je jedan od \v{c}estih
slu\v{c}ajeva kad nas vjeroajatnosni osje\'{c}aj vara. (Prosjeci i
vjerojatnost nisu isto). Zaboravimo zato osje\'{c}aje i krenimo
ra\v{c}unati. Vjerojatnost od $x$ \v{s}estica u $n$ bacanja je
\[\binom{n}{x}\left(\frac{1}{6}\right)^x \left(\frac{5}{6}\right) ^{n-x} \;\;\;\; x=0,1,\ldots ,n \]
Kada bacimo $6n$ kocaka vjerojatnost od $n$ ili vi\v{s}e
\v{s}estica je
\[ \sum _{x=n}^{6n} \binom{n}{x}\left(\frac{1}{6}\right)^x\left(\frac{5}{6}\right)^{n-x}\]

Newton je izra\v{c}unao ove vrijednosti "ru\v{c}no", a mi se danas
mo\v{z}emo poslu\v{z}iti tablicama kumulativne binomne
distribucije, ili ra\v{c}unalnim paketima koji ih sadr\v{z}e, i
dobit \'{c}emo sljede\'{c}e vrijednosti:

\begin{center}
\begin{tabular}{c|c|c}
$n$ & $6n$ & $P(\geq n$ \v{s}estica$)$\\
\hline 1 & 6 & 0.665\\
2 & 12 & 0.619 \\
3 & 18 & 0.597 \\
4 & 24 & 0.584 \\
5 & 30 & 0.576 \\
\vdots & \vdots & \vdots \\
100 & 600 & 0.517
\end{tabular}
\end{center}

Dakle, najvjerojatnija je barem 1 \v{s}estica u 6 bacanja, a
najmanje vjerojatne su barem 3 \v{s}estice u 18 bacanja. To je
bilo i Newtonovo rje\v{s}enje. Samuel Peppys je doznao na \v{s}to
se treba kladiti.

\item Ve\'{c}ina ljudi ima osje\'{c}aj da je u skupini
veli\v{c}ine \v{s}kolskog razreda vjerojatnost podudarnog ro\dj
endana puno manja od $50\%$. I to je jedan od \v{c}estih
slu\v{c}ajeva lo\v{s}eg osje\'{c}aja za vjerojatnost. Naime, u
skupini te veli\v{c}ine podudarnost je oko $70\%$. Okrenimo se
opet ra\v{c}unu. Neka je $N=365$ broj jednako vjerojatnih dana
(zanemarit \'{c}emo prijestupne godine), a $s$ broj ljudi u
promatranoj skupini. Izra\v{c}unajmo vjerojatnost svih da svih $s$
ljudi ima me\dj usobno razli\v{c}ite ro\dj endane.

Ro\dj endan prvoga od $s$ ljudi mo\v{z}e biti bilo kojeg od $N$
dana. Drugi mora biti razli\v{c}it, pa zato mo\v{z}e tek biti bilo
koji od preostalih $N-1$ dana. Tre\'{c}i mo\v{z}e biti bilo koji
od preostalih $N-2$, itd., do preostalih $N-s+1$ dana. Dakle,
ukupan broj na\v{c}ina na koji se mogu realizirati svi me\dj
usobno razli\v{c}iti ro\dj endani je
\[N(N-1)(N-2)\ldots (N-s+1)=\frac{N!}{(N-s)!}.\]

S druge strane, ukupan broj na\v{c}ina na koji se mogu realizirati
svi mogu\'{c}i ro\dj endani je  $N\cdot N\cdot N\cdot \ldots \cdot
N=N^s$.

Odavde slijedi da vjerojatnost svih me\dj usobno razli\v{c}itih
ro\dj endana iznosi
\[\frac{N}{N}\frac{N-1}{N}\frac{N-2}{N}\ldots \frac{N-s+1}{N}=\frac{N!}{(N-s)! N^s} \]

Komplementarna vjerojatnost,
\[P_s=1-\frac{N!}{(N-s)!N^s} ,\]
je vjerojatnost da skupina od $s$ ljudi ima barem jedan podudarni
ro\dj endan.U sljede\'{c}oj tablici dani su iznosi te
vjerojatnosti za $N=365$ i neke $s$.

\begin{center}
\begin{tabular}{|c|c|c|c|c|c|c|c|}
\hline $s$ & 5& 10 & 20 & 23 & 30 & 40 & 60\\
\hline $P_s$ & 0.027 & 0.117 & 0.411 & 0.507 & 0.706 & 0.891 &
0.994\\
\hline
\end{tabular}
\end{center}

Vidimo da je vjerojatnost podudarnog ro\dj endana ve\'{c}a od
$50\%$ ve\'{c} u skupini od 23 \v{c}ovjeka. U skupini od 60 ljudi
podudaran ro\dj endan gotovo je siguran.

Najte\v{z}i problem u gornjemu izvodu je ra\v{c}unanje vrijednosti
$P_s$. Naime, $365!$, $(365-23)!$ i $365^{23}$ ogromni su brojevi
koji zahtijevaju ogromne ra\v{c}une. Zato te ra\v{c}une ovdje
nismo ni uveli, nego smo (u tablici) dali samo njihove kona\v{c}ne
rezultate.

Puno je lak\v{s}e provesti pribli\v{z}ne ra\v{c}une, poput
sljede\'{c}eg koji se koristi dobro poznatom aproksimacijom:
\[e^{-x}=1-x +\frac{x^2}{2!}-\frac{x^3}{3!}+\frac{x^4}{4!}-\ldots \approx 1-x .\]

Ta je aproksimacija dobra za male vrijednosti $x$ i obi\v{c}no se
koristi tako da desnom stranom $1-x$ aproksimiramo lijevu stranu
$e^{-x}$. Mi \'{c}emo je koristiti u obrnutom smjeru.

\begin{equation*}
\begin{split}
\frac{N!}{(N-s)!N^s} &=\frac{N}{N}\frac{N-1}{N}\frac{N-2}{N}\ldots
\frac{N-s+1}{N}=\\
&=\Bigl(1-\frac{1}{N}\Bigr)\Bigl(1-\frac{2}{N}\Bigr)\ldots\Bigl(1-\frac{s-1}{N}\Bigr)\approx\\
& \approx e^{-\frac{1}{N}}e^{-\frac{2}{N}}\ldots
e^{-\frac{s-1}{N}}=e^{-\frac{1+2+\ldots +(s-1)}{N}}=\\
&= e^{-\frac{s(s-1)}{2N}}
\end{split}
\end{equation*}

Uo\v{c}imo da je ta aproksimacija (zbog $e^{-x}>1-x$) ve\'{c}a od
tra\v{z}ene vrijednosti. Sada lako zaklju\v{c}ujemo da je
\[P_s=1-\frac{N!}{(N-s)!N^s}\approx 1-e^{-\frac{s(s-1)}{2N}}\]

i da je ta aproksimacija manja od tra\v{z}ene vrijednosti.
\v{Z}elimo li na\'{c}i $s$ za koji je $P_s>0.5$, valja nam
rije\v{s}iti jednad\v{z}bu
\[1-e^{-\frac{s(s-1)}{2\cdot 365}}=0.5, \;\;\; e^{-\frac{s(s-1)}{830}}=0.5, \;\;\; \frac{-s(s-1)}{830}=\ln{0.5},\]
\[s^2 -s + 830 \ln{0.5}=0.\]

Ta kvadratna jednad\v{z}ba ima pozitivno rje\v{s}enje $24<s<25$.
Dakle, za $s=25$ sigurno vrijedi $P_s>0.5$. S obzirom na to da su
na\v{s}e aproksimacije bile manje od tra\v{z}ene vrijednosti,
mo\v{z}da je $P_s >0.5$ i za $s=23$ ili \v{c}ak $s=22$. To ovaj
ra\v{c}un s $e^{-x}$ ne mo\v{z}e odlu\v{c}iti. Me\dj utim,
to\v{c}niji rezultati pokazuju da je $s=23$ prvi $s$ za koji je
$P_s>0.5$.

\end{enumerate}


%%%%%%%%%%%%%%%%%%%%%%%%%%%%%%%%%%%%%%

\newpage
{\scriptsize \listoffixmes }


\end{document}
